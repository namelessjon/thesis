\subsection{Activation of the Heart}

In normal function the heart depolarises in a rhythmic and controlled fashion,
at a rate of 60 to 100 bpm.
There exist numerous abnormal rhythms, or arrhythmias, which are diagnosed in
patients with heart disease.
These can be divided into two broad categories.
Slowed heart rate, or bradycardia, with a heart rate below 60 bpm and quickened
heart rate, tachycardia, with a sustained heart rate over 100 bpm.
Bradycardia and tachycardia manifest in differing ways in the atria and
ventricles; this section focuses on atrial manifestations.

\subsubsection{Normal Activation of the Heart}

The normal heartbeat originates in the right atrium, in the sino-atrial node.
This region of auto-active cells regularly depolarises, sending out excitation
waves through the cardiac tissue.
These excitation waves spread in all directions along the atrial walls, though
they are preferentially conducted along the crista terminalis to another area of
specialised cells, the atrio-ventricular node.
The atrio-ventricular node slows conduction through it, due to small and low
capacitance cells.

While the atrio-ventricular node is depolarising, the atria finish their own
depolarisation.
In the right atrium, this spreads from the sino-atrial node and the fast
conducting muscle ridges of the crista terminalis and pectinate muscles.
Excitation is conducted to the left atrium through the Bachmann bundle or
an inferior muscle bundle.
The atrial depolarisation leads to the atria contracting and charging the
ventricles with blood.

The atrio-ventricular node depolarising carries the excitation through the
central fibrous body and into the bundle of His.
The bundle of His is made of fast conducting muscle fibres and soon splits into
the left and right bundle branches, which form the start of the purkinje fibre
network.
The purkinje fibres are made of another specialised cell type, and they conduct
the electrical signal quickly through the body of the ventricular muscle.
The purkinje fibres break out on the endocardial surface of the ventricles, the
only place they are electrically coupled to the normal ventricular muscle.

As the excitation wave emerges from the breakout point, it rapidly conducts
through the thickness of the ventricular wall to the epicardial surface.
This creates a powerful contraction which forces the blood from the ventricles.
The ventricles then depolarise, as a result of the complex action potential
heterogeneity in the ventricular walls, in much the same direction; from the
inside out.

\subsubsection{Mechanisms of Bradycardia}

In the atrium, bradycardia, slow heart rate, is most commonly due to increased
parasympathetic (or vagal) tone.
This can actually be `normal' in the case of athletes, whose training can result
in them having very low resting heart rate.
Other causes can include cardiac drugs, such as beta blockers, or hypothyroid.

Sick sinus syndrome is a grouping of conditions, most common in the elderly.
In conditions of sick sinus syndrome, the bradycardia is pronounced enough to be
dangerous.
As a grouping of conditions, sick sinus syndrome has a number of causes.
Congenital sick sinus syndrome has been linked to mutations of the SCN5A
protein~\cite{Benson2003}, which forms part of the sodium channels in the sinus
node.
In elderly patients, the cause can be more varied.
The condition has been linked to increased fibrosis of cardiac
tissue~\cite{Kohl2005}\ and also due to age related changes in sinus node
electrophsyiology~\cite{Alings1993}.

\subsubsection{Mechanisms of Tachycardia}

In the atrium, tachycardia, fast heart rate, is the sustained heart rate of over
100 bpm.
In cases of atrial flutter, the rate is typically around 300 bpm, although the
atria are still contracting normally, if very quickly.
In atrial fibrillation, stimulation rates of 400 bpm or higher can be observed
and in addition, the electrical excitation is chaotic and so the atria don't
contract effectively.
Flutter and fibrillation can arise from a number of sources, but the most common
two are re-entry and ectopic foci~\cite{Nattel2002,Wyse2004,Mandapati2000,Haissaguerre1998,Tsai2000,Prystowsky2008}.

Re-entry occurs when one stimulus gives rise to two (or more) propagations of
excitation.
The extra propagations can arise from a number of reasons.
They can also become anchored to a particular point, forming a stable rotor and
spiral waves.

The most common cause of re-entry is due to unidirectional conduction block.
Unidirectional conduction block is a region of tissue which conducts excitation
waves in only one direction.
Unidirectional conduction block can arise from a number of reasons.
These include injury, such as a scar from cardiac surgery or ischaemic injury
caused by a blockage of blood vessels within the heart.
Unidirectional conduction becomes important when electrical excitation can
travel along two separate paths, for example around the openings of the
tricuspid or mitral valves, down the crista terminalis or between the atria via
the Bachmann's bundle and inferior pathways.
If there is unidirectional conduction block within such a region, the wave will
not propagate along that path.
This potentially allows for conduction along the other path to loop back to the
first path which will still be excitable.
Re-entry occurs when the timing of such a retrograde conduction means that the
excitation wave exits the first pathway when the cells beyond are out of the
refractory period.
Excitation can then propagate again, potentially many times in a sustained
re-entry.

Regions of unidirectional block can also arise dynamically.
This occurs due to heterogeneous restitution properties or slowed conduction.
Due to different adaptations to pacing, one region can still be excited
when a neighbouring region is no longer refractory.
When this occurs the neighbouring region can become excited, and so a re-entry
occurs.
This is a spiral wave, when it occurs way from an anatomical obstacle as it
tends to result in the excitation wave spiraling around the region of slowed
conduction.

Ectopic foci are sources of excitation which are not the sino-atrial node.
These can result from after-depolarisations.
They can also be caused by other auto-active cells, most commonly associated
with the pulmonary vein region~\cite{Haissaguerre1998}.

After-depolarisations arise principally from interactions within the calcium
handling of the cell, particularly the Sodium--Calcium exchanger.
The exchanger is responsible for expelling Calcium from the cell.
Since one Calcium ion is expelled for every three Sodium ions brought within,
this generates an inward current.
This inward current can result in the threshold potential being reached and so
initiating another action potential.
This action potential can then spread, if conditions are favourable, in a manner
similar to re-entry.

Other auto-active cells are associated most commonly with the pulmonary vein
region, although the superior vena cava has also been noted in some
studies~\cite{Tsai2000}.
This is most typically observed in rapid bursts with a rate of over 300 bpm.
Since it has a higher rate than the atrium, it becomes the dominant pacemaker
within the atrium, suppressing normal sinus rhythm.
The exact mechanisms for these rapid bursts of pacing are still under
investigation~\cite{Chen2000}.

Once tachycardia has been initiated there is evidence for what is termed
`electrophysiological remodelling'~\cite{Stott2008,Workman2001,Bosch1999}.
Electrophysiological remodelling is the alteration of cellular electrical
properties caused by the more rapid atrial activity.
This remodelling has been shown to reduce action potential duration and
effective refractory period; thus it can increase the susceptibility of atrial
tissue to further tachycardia.



