\chapter{The Body Surface Potential}

Whilst modelling the heart itself can provide valuable insights into the effects
of diseases and inherited conditions, as Chapters 2 and 3 show, a clinical
doctor cannot look at the heart's activity directly without a surgical
procedure.
Instead they must rely on external tools such as the echocardiogram and the
electro-cardiogram (ECG).
To reproduce the ECG with mathematical models, it is necessary to solve what is known as
the ``forward problem''.


\section{The Forward Problem}

To solve the forward problem, Maxwell's equations must be solved to determine the
field in the torso which arises from currents flowing within the heart.
Due to the nature of the problem---the finite size of the torso and the
relatively low frequencies involved---simplifying assumptions can be made.
The effects of propagation and of capacitive and inductive currents may be
neglected~\cite{Barnard1966}.
The situation must solve therefore becomes a quasi-static volume conductor
problem, which involves only tissue conductances.
The current flow in the torso, $\mathbf{J}$, is given by Ohm's law
\begin{equation}
\label{bsp:ohm}
\mathbf{J} = \sigma\mathbf{E} + \mathbf{J}^{i}
\end{equation}
where $\mathbf{E}$ is the electric field, $\sigma$ is the tissue conductivity
and $\mathbf{J}^{i}$ is an impressed, or applied, current.
The applied current term is included to allow for the presence of active sources and
is non-zero only at the locations of active sources, i.e. the heart.
Since the total current in (\ref{bsp:ohm}) is solenoidal (the net flow into or
out of any closed region is zero),
\begin{equation}
\label{bsp:ohm2}
\nabla \cdot \mathbf{J} = 0 = \sigma \nabla \cdot \mathbf{E} + \nabla \cdot \mathbf{J}^{i}
%= 0 = \sigma \bigtriangledown \cdot \mathbf{E} +
%    \bigtriangledown \cdot \mathbf{J}^{i}
\end{equation}
must be true.
As we are solving the quasi-static problem, $\mathbf{E}$ can be
found simply from the gradient of the scalar potential, $\phi$, as
\begin{equation}
\label{bsp:maxwell}
\mathbf{E} = - \nabla\phi
\end{equation}

If (\ref{bsp:maxwell}) is substituted into (\ref{bsp:ohm2}) then we obtain
\begin{equation}
\label{bsp:poisson}
\nabla^{2}\phi = \frac{\nabla \cdot \mathbf{J}^{i}}{\sigma}
\end{equation}
which can be recognised as Possion's equation.
In an infinite homogeneous conducting medium, a solution to Poisson's equation
for the field at any given point, $\phi\left(x\right)$, is~\cite{Plonsey1963}
\begin{equation}
\label{bsp:infinite}
\phi\left(\mathbf{x}\right) = \frac{1}{4 \pi \sigma} \int \frac{- \nabla \cdot
\mathbf{J}^{i} }{\mathbf{r}} dV
\end{equation}
where $\mathbf{r}$ is a vector from source point $\mathbf{x'}$ to field point $\mathbf{x}$.
To account for the influence of the finite dimensions of the torso we can use
either a FEM or a BEM (See Chapter XXXX for a discussion
of the history of the two methods).
The derivation for the BEM method is based on Green's Theorem~\cite{Barr1966},
which states that for a volume, $V$, bounded by a surface, $S$, that
\begin{equation}
\label{bsp:green}
\int_{V} \left(\phi \nabla^{2}\psi - \psi \nabla^{2}\phi  \right) dV =
\int_{S} \left( \phi \nabla \psi - \psi \nabla \phi \right) \cdot d\mathbf{S}
\end{equation}
where $\phi$ and $\psi$ are scalar functions of position.
If $\phi$ is the electrical potential and $\psi$ is set as $\frac{1}{r}$ where
$r$ is $|\mathbf{r}|$, the same $\mathbf{r}$ as in (\ref{bsp:infinite}).

