\chapter{The Body Surface Potential}

Whilst modelling the heart itself can provide valuable insights into the effects
of diseases and inherited conditions, as Chapters 2 and 3 show, a clinical
doctor cannot look at the heart's activity directly without a surgical
procedure.
Instead they must rely on external tools such as the echocardiogram and the
electro-cardiogram (ECG).
To reproduce the ECG with mathematical models, it is necessary to solve what is known as
the ``forward problem''.


\section{The Forward Problem}

To solve the forward problem, Maxwell's equations must be solved to determine the
field in the torso which arises from currents flowing within the heart.
Due to the nature of the problem---the finite size of the torso and the
relatively low frequencies involved---simplifying assumptions can be made.
The effects of propagation and of capacitive and inductive currents may be
neglected~\cite{Barnard1966}.
The situation must solve therefore becomes a quasi-static volume conductor
problem, which involves only tissue conductances.
The current flow in the torso, $\mathbf{J}$, is given by Ohm's law
\begin{equation}
\label{eqn:bsp:ohm}
\mathbf{J} = \sigma\mathbf{E} + \mathbf{J}^{i}
\end{equation}
where $\mathbf{E}$ is the electric field, $\sigma$ is the tissue conductivity
and $\mathbf{J}^{i}$ is an impressed, or applied, current.
The applied current term is included to allow for the presence of active sources and
is non-zero only at the locations of active sources, i.e. the heart.
Since the total current in (\ref{eqn:bsp:ohm}) is solenoidal (the net flow into or
out of any closed region is zero),
\begin{equation}
\label{eqn:bsp:ohm2}
\nabla \cdot \mathbf{J} = 0 = \sigma \nabla \cdot \mathbf{E} + \nabla \cdot \mathbf{J}^{i}
%= 0 = \sigma \bigtriangledown \cdot \mathbf{E} +
%    \bigtriangledown \cdot \mathbf{J}^{i}
\end{equation}
must be true.
As we are solving the quasi-static problem, $\mathbf{E}$ can be
found simply from the gradient of the scalar potential, $\phi$, as
\begin{equation}
\label{eqn:bsp:maxwell}
\mathbf{E} = - \nabla\phi
\end{equation}

If (\ref{eqn:bsp:maxwell}) is substituted into (\ref{eqn:bsp:ohm2}) then we obtain
\begin{equation}
\label{eqn:bsp:poisson}
\nabla^{2}\phi = \frac{\nabla \cdot \mathbf{J}^{i}}{\sigma}
\end{equation}
which can be recognised as Poisson's equation.
In an infinite homogeneous conducting medium, a solution to Poisson's equation
for the field at any given point, $\phi$, is~\cite{Plonsey1963}
\begin{equation}
\label{eqn:bsp:infinite}
\phi = \frac{1}{4 \pi \sigma} \int \frac{- \nabla \cdot
\mathbf{J}^{i} }{r} dv
\end{equation}
where $r$ is the (scalar) distance from the elemental volume $dv$ to the point at
which the field is being evaluated.

To account for the influence of the finite dimensions of the torso we can use
either a FEM or a BEM (See Chapter XXXX for a discussion
of the history of the two methods).
The derivation for the BEM method is based on Green's Theorem~\cite{Barr1966},
which states that for a volume, $V$, bounded by a surface, $S$, that
\begin{equation}
\label{eqn:bsp:green}
\int_{V} \left(\phi \nabla^{2}\psi - \psi \nabla^{2}\phi  \right) dv =
\int_{S} \left( \phi \nabla \psi - \psi \nabla \phi \right) \cdot d\mathbf{S}
\end{equation}
where $\phi$ and $\psi$ are scalar functions of position.
If $\phi$ is the electrical potential and $\psi$ is set as $\frac{1}{r}$ where
$r$ is $|\mathbf{r'}-\mathbf{r}|$.
Here, $\mathbf{r'}$ is a vector to an arbitrary point in the volume $V$ at which
we wish to evaluate the field and $\mathbf{r}$ is a vector to an elemental
volume, $dv$, somewhere within the volume $V$.
Using (\ref{eqn:bsp:poisson}) we have
\begin{equation}
\label{eqn:bsp:greenandpoisson}
\int_{V}
    \left(
        \phi \nabla^{2}\left(\frac{1}{r}\right) -
        \frac{1}{r} \nabla^{2}\frac{\left(\nabla \cdot \mathbf{J}^{i} \right)}{\sigma}
    \right)
dv =
\int_{S}
    \left(
        \phi \nabla \left(\frac{1}{r}\right) -
        \left(\frac{1}{r}\right) \nabla \phi
    \right)
\cdot d\mathbf{S}
\end{equation}

The del operator in (\ref{eqn:bsp:greenandpoisson}) operates on the unprimed (source) coordinates.
Now,
\begin{equation}
\label{eqn:bsp:oneoverr}
\nabla^{2}\left(\frac{1}{r}\right) =
\nabla^{2}\left(\frac{1}{|\mathbf{r'}-\mathbf{r}|}\right) =
-4\pi\delta\left(\mathbf{r'}-\mathbf{r}\right)
\end{equation}
where $\delta$ represents the dirac delta function.  The surface $S$ is the body
surface and so on $S$, $\nabla\phi \cdot d\mathbf{S} = 0$ to a very good
approximation.  (\ref{eqn:bsp:greenandpoisson}) becomes, after substitution and
rearrangement,
\begin{equation}
\label{eqn:bsp:substituted}
\phi\left(\mathbf{r'}\right) =
\frac{1}{4 \pi \sigma}\int_{V} \frac{\nabla \cdot \mathbf{J}^{i}}{r}dv - 
\frac{1}{4 \pi}\int_{S} \phi\left(\mathbf{r}\right)
\nabla\left(\frac{1}{r}\right) \cdot d\mathbf{S}
\end{equation}

By noting that
\begin{equation}
\label{eqn:bsp:solidanglesubs}
-\nabla\left(\frac{1}{r}\right) \cdot d\mathbf{S} =
-\frac{\left(\mathbf{r'}-\mathbf{r}\right)}{|\mathbf{r'}-\mathbf{r}|^{3}} \cdot d\mathbf{S} =
d\Omega
\end{equation}
where $d\Omega$ is a differential element of solid angle,
(\ref{eqn:bsp:substituted}) becomes
\begin{equation}
\label{eqn:bsp:substitutedomega}
\phi\left(\mathbf{r'}\right) =
\frac{1}{4 \pi \sigma}\int_{V} \frac{\nabla \cdot \mathbf{J}^{i}}{r}dv +
\frac{1}{4 \pi}\int_{S} \phi\left(\mathbf{r}\right)d\Omega
\end{equation}
The first term on the right hand side can be recognised as the infinite medium
potential (\ref{eqn:bsp:infinite}) and the second term consists of contributions
from the torso surface.
To discretise (\ref{eqn:bsp:substitutedomega}) we can consider $S$ to be made up of
$n$ triangles, leading to
\begin{equation}
\label{eqn:bsp:discrete}
\phi\left(\mathbf{r'}\right) \approx
\frac{1}{4 \pi \sigma}\int_{V} \frac{\nabla \cdot \mathbf{J}^{i}}{r}dv +
\frac{1}{4 \pi}\sum_{j=1}^n \phi_{j}\Delta\Omega_{j}
\end{equation}
where $\phi_{j}$ is the potential on the j\textsuperscript{th}\ surface element
and $\Delta\Omega_{j}$ is the increment of solid angle of the
j\textsuperscript{th}\ element when viewed from $\mathbf{r'}$.
To find a solution, Barr et al. noted that $\phi\left(\mathbf{r'}\right)$ is the
potential at an arbitrary point inside $V$.
If these points are chosen to be at the centres of the triangles just inside
the surface $S$ then since $\nabla\phi \cdot d\mathbf{S} = 0$ we can get an
expression for the potential on the i\textsuperscript{th}\ triangle, $\phi_{i}$,
\begin{equation}
\label{eqn:bsp:discretei}
\phi_{i} =
\frac{1}{4 \pi \sigma}\int_{V} \frac{\nabla \cdot \mathbf{J}^{i}}{r}dv +
\frac{1}{4 \pi}\sum_{j=1}^n \phi_{j}\Delta\Omega_{ji}
\end{equation}
where $\Delta\Omega_{ji}$ is the solid angle of the j\textsuperscript{th}\
triangle seen from the i\textsuperscript{th}\ triangle.
In the summation in (\ref{eqn:bsp:discretei}) there is one term which corresponds to
the case where $i = j$.
In this case, $\Delta\Omega_{ii} = 2\pi$ as from a point just inside $i$, $i$
will obscure an angle of $2\pi$.
Equation (\ref{eqn:bsp:discretei}) then becomes, after rearrangement,
\begin{equation}
\label{eqn:bsp:discretefinal}
\frac{\phi_{i}}{2} + \sum_{j=1,j \neq i}^n \left(-\frac{\Delta\Omega_{ji}}{4\pi} \right)\phi_{j} =
\frac{1}{4 \pi \sigma}\int_{V} \frac{\nabla \cdot \mathbf{J}^{i}}{r}dv
\end{equation}
which represents a set of $n$ simultaneous equations for the potentials on the
surface elements of the torso.
Taking
\begin{equation}
\label{eqn:bsp:b}
B_i = \frac{1}{4 \pi \sigma}\int_{V} \frac{\nabla \cdot \mathbf{J}^{i}}{r}dv
\end{equation}
equation (\label{eqn:bsp:discretefinal}) can be written in matrix form as
\begin{equation}
\label{eqn:bsp:matrix}
\mathbf{A}\mathbf{\phi} = \mathbf{B}
\end{equation}
where $\mathbf{A}$ is a matrix which depends entirely on the geometry of the
torso surface with a typical term of $\displaystyle A_ij =
-\frac{\Delta\Omega_ji}{4\pi}$ and $A_ii = 0.5$, $\mathbf{\phi}$ is a column
vector of the potentials of the $n$ triangles and $\mathbf{B}$ is a column
vector of the infinite medium potentials at the centres of the triangles of the
surface.

If the surface $S$ was discretised with sufficient accuracy then $\mathbf{A}$
will necessarily be singular, due to the physical nature of the problem.
This was noted by Salu~\ref{Salu1980}\ who proposed a solution which takes
advantage of the physical properties of the system (an alternative method of
removing the singularity of the system was proposed by Lynn and Timlake, the
deflation method).
Salu noted that, experimentally, the potential $\phi$ can only be determined up to
an additive constant.
Therefore $\phi$ can be taken as $0$ at arbitrarily chosen point, without
effecting the general solution.
Assigning $\phi_1 = 0$ in (\ref{eqn:bsp:matrix}) leads to
\begin{equation}
\label{eqn:bsp:salufirst}
\sum_{j=2}^n A_ij \phi_j = B_i \quad\quad  i = 1,\cdots, n
\end{equation}
which is a set of $n$ equations in $n-1$ unknowns.
These equations should have exactly one solution.
If an exact solution exists, this implies two things: (a)
(\ref{eqn:bsp:salufirst}) is a set of $n$ consistent equations in $n-1$ unknowns
and (b) the rank of the sub-matrix $A_ij: i=2,\cdots,n j=2,\cdots,n$ is $n-1$.
Hence there are $n$ nontrivial $\lambda_i$ such that the rows of $\textbf{A}$
fulfil
\begin{equation}
\label{eqn:bsp:salulambda}
\sum_{i=1}^n \lambda_i A_ij  = 0 \quad\quad  j = 2,\cdots, n
\end{equation}
The $\lambda_i$s may be determined up to a proportional factor.
For (\ref{eqn:bsp:matrix}) to be consistent, it is also required that
\begin{equation}
\label{eqn:bsp:salulambdaB}
\sum_{i=1}^n \lambda_i B_i  = 0
\end{equation}
Salu notes that (\ref{eqn:bsp:salulambdaB}) might not hold for a number of
reasons, including numerical inaccuracies in the discretisation of surface or
errors in the calculation of the $B_i$s.
This would lead to a difference between $\phi_{\text{calculated}}$ and
$\phi_{\text{real}}$.
Considering once more the physical properties of the system, $B_i$ as an
electrostatic potential could only be determined up to some additive constant,
$\alpha$.
Equation (\ref{eqn:bsp:salulambda}) then becomes
\begin{equation}
\label{eqn:bsp:salulambdaalpha}
\sum_{i=1}^n \lambda_i \left(B_i+\alpha\right)  = 0
\end{equation}

This can be incorporated into (\ref{eqn:bsp:salufirst}) to get a set of $n$
equations
\begin{equation}
\label{eqn:bsp:salufinal}
\sum_{j=2}^n A_ij \phi_j = B_i + \alpha \quad\quad  i = 1,\cdots, n
\end{equation}
Equation (\ref{eqn:bsp:salufinal}) is now numerically consistent and is
equivalent to (\ref{eqn:bsp:matrix}).
Whenever (\ref{eqn:bsp:salulambdaB}) is not fulfilled, due to numerical errors
in the discritisation and creation of $\textbf{A}$ or in the calculation of
$\textbf{B}$, the addition of the $\alpha$ term ensures that
(\ref{eqn:bsp:salufinal}) has a consistent solution.
It is important to note that consistent does not mean necessarily mean accurate
or correct.
The addition of the $\alpha$ term merely ensures that a solution will exist.
Due care must still be taken with the construction of each $A_ij$ term and
accuracy can be improved via choosing a finer discretisation for the body
surface mesh.

An efficient solution to the problem of solving the equations and for $\alpha$
was given by Salu.
To do this, we let $\phi_j^* \left(j=2,\cdots,n\right)$ be a solution to the
$n-1$ equations
\begin{equation}
\label{eqn:bsp:saluphijstar}
\sum_{j=2}^n A_ij \phi_j^* = B_i \quad\quad  i = 2,\cdots, n
\end{equation}
and let $\phi_j^1 \left(j=2,\cdots,n\right)$ be a solution to the $n-1$
equations
\begin{equation}
\label{eqn:bsp:saluphijone}
\sum_{j=2}^n A_ij \phi_j^1 = 1 \quad\quad  i = 2,\cdots, n
\end{equation}
where the $1$ represents a column vector of 1s.
The two vectors $\phi_j^*$ and $\phi_j^1$ both multiply the same matrix,
$\mathbf{A}$ so it need only be inverted once to solve both
(\ref{eqn:bsp:saluphijstar}) and (\ref{eqn:bsp:saluphijone}).
Letting $\phi_j \left(j=2,\cdots,n\right)$ be a solution to the set of $n-1$
equations
\begin{equation}
\label{eqn:bsp:saluphij}
\sum_{j=2}^n A_ij \phi_j = B_i + \alpha \quad\quad  i = 2,\cdots, n
\end{equation}
where $\alpha$ is the same alpha introduced in (\ref{eqn:bsp:salulambdaalpha}).
From equations (\ref{eqn:bsp:saluphijstar})--(\ref{eqn:bsp:saluphij}), the
solution to the set of equations will be
\begin{eqnarray}
\label{eqn:bsp:salusolution}
\phi_j&=&\phi_j^* + \alpha\phi_j^1 \quad\quad  j = 2,\cdots, n \nonumber\\
\phi_1&=&0
\end{eqnarray}

Substituting (\ref{eqn:bsp:salusolution}) into the first equation of
(\ref{eqn:bsp:salufinal}) will give
\begin{equation}
\label{eqn:bsp:salualphastep1}
\sum_{j=2}^n A_1j \left(\phi_j^* + \alpha\phi_j^1\right) = B_i + \alpha
\end{equation}
or after re-arranging to solve for $\alpha$
\begin{equation}
\label{eqn:bsp:salualphastep2}
\alpha = 
    \frac{
        \left(\sum_{j=2}^n A_1j \phi_j^*\right)-B_i
        }{
        1-eft(\sum_{j=2}^n A_1j \phi_j^1
        }
\end{equation}
Equation (\ref{eqn:bsp:salualphastep2}) can be used along with
(\ref{eqn:bsp:saluphijstar})--(\ref{eqn:bsp:saluphij}) to solve
(\ref{eqn:bsp:matrix}) to find the body surface potential.

\section{The Atrial Dipole}

In the previous section, the method for solving Maxwell's equations to determine
the body surface potential was derived.
The $\mathbf{B}$ term in the equation (\ref{eqn:bsp:matrix}) is the infinite
homogeneous medium potential for a number of impressed current sources
$\mathbf{J}^i$.
The model of atrial electrophysiological activity developed in the previous
chapter provides an output of the trans-membrane potentials, $V_m$.
To relate the values of $V_m$ to $\mathbf{J}^i$ we go back to dipole theory and
a definition of 

