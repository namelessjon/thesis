\chapter{Applications Of The Forward Problem}

The ECG is the first tool cardiac doctors turn to when diagnosis of a problem is
required.
ECG machines can be found in almost every hospital in the world.
A model of the atria and the surface potentials developed by the excitation of
the model can be used to guide diagnosis of a variety of conditions.
This can reduce the need for surgical procedures or suggest when they are
essential.

Whilst the inverse problem promises to reproduce the potentials on the heart
from the potentials on the surface, the technique has limitations.
For accurate solutions, patient specific geometries have to be constructed from
MRI scans.
There is also a need for complex lead systems, sometimes featuring more than two
hundred leads.
Also, many of the inverse techniques rely on `smooth' propagation patterns to
reduce the uncertainties in the technique which may not be found in pathological
cases.
A device which can perform such calculations automatically is a long way off,
both in terms of computational power required and complexities to resolve.

By contrast, diagnostic guides based on a forward solution can be of use to any
doctor.
They can also be used to further validate simulation studies of genetic or
diseased conditions, by comparison of the generated ECGs with those recorded
from real patients.
This chapter explores some of these predictions, using the model developed in
the previous chapter.

\section{Inverted P-Waves at Night}

Recently an observation was made~\cite{BoyettPrivate}\ in patients under 24 hour
ECG monitoring.
It was noted that some patients exhibited inverted P-waves at night.
That is to say, if the patient showed a upright P-wave in leads II and aVF
during the day, then at night the P-wave would be downwards in leads II and aVF.
This phenomena has not been reported in the literature.

There is evidence~\cite{Shibata2001,Boineau1988} that the pacemaker is not a
small and discrete area of the atrium, but is instead distributed along the
length of the crista terminalis.
The presence of certain drugs and hormones, most notably astylcholin, can cause
the site of the leading pacemaker to move down the pace maker complex.
Astylcholin is released by what is know as increased `vagal tone'.
This has been observed to happen at night.
It was hypothesised that a pacemaker shift induced by this increased vagal tone
might lead to the observed P-wave inversion.

\subsection{Methods}

In the absence of a model for the distributed pacemaker complex in the human
heart, the direct effects of astylcholin could not be investigated.
Instead, using the model presented in the previous chapter, several sites were
located along the crista terminalis.
These sites had a radius of 15 nodes (or approximately \mm{5}--although this
varied depending on the thickness of the atrial wall at the pacing site), and
therefore were approximately the same size as the sino-atrial node.
These sites are shown in Figure~\ref{fig:forward:ct_sites}.
Each of these sites was stimulated via the same protocol used to stimulate the
sinus node in the original model and then the electrical excitation waves were
allowed to propagate without interference.

ECGs were computed from the patterns of electrical excitation in the atrium.
These were compared to the sinus rhythm ECGs computed in the previous chapter.
In addition, using a so called `inverse Dower' method after Edenbrandt and
Pahlm~\cite{Edenbrandt1988}, the orthogonal components of the ECG were computed
and used to construct representations of the heart
vector~\cite{Frank1956,MacFarlane1989a} ECG (VECG).
To perform the inverse dower transformation, a matrix which was optimized for
the P-wave (shown in Table~\ref{tbl:forward:idparams}) was
used~\cite{Guillem2007}.


\begin{table}
\caption[Inverse Dower Factors]{
\label{tbl:forward:idparams}
Factors to construct the Frank VECG from the standard 12 lead ECG set.
Parameters optimised to accurately reproduce the P-wave heart
vector~\cite{Guillem2007}.
Each of the 8 leads are multiplied by the given parameters to provide the
orthogonal Frank lead.
}
\begin{center}
\begin{tabular}{c c c c c c c c c}
\toprule
& $\text{V}_{\text{1}}$ &$\text{V}_{\text{1}}$ & $\text{V}_{\text{1}}$ & $\text{V}_{\text{1}}$ & $\text{V}_{\text{1}}$ & $\text{V}_{\text{1}}$ & I && II \\
\midrule
X & $-0.266$ & $\:0.027$ &  $\:0.065$ & $\:0.131$ & $\:0.203$ & $\:0.220$ & $\:0.370$ & $-0.154$ \\
Y & $\:0.088$ &  $-0.088$ & $\:0.003$ & $\:0.042$ & $\:0.047$ & $\:0.067$ & $-0.131$ & $\:0.717$ \\
Z & $-0.319$ & $-0.198$ & $-0.167$ & $-0.099$ & $\:0.009$ & $\:0.060$ & $\:0.184$ & $\:0.114$ \\
\bottomrule
\end{tabular}
\end{center}
\end{table}

\subsection{Results}





\subsection{Limitations}

Edenbrandt and Pahlm verses Uijen, optimised sets.  Perhaps Hyttinen paper.
Optimised for ST segment, not P-wave.
The original inverse dower matrix, as presented by Edenbrandt and
Pahlm~\cite{Edenbrandt1988}\ was used to perform the transformations, as used in
previous studies~\cite{Carlson2005,Holmqvist2007,Havmoller2007}

\section{Focal Atrial Tachycardia}

Atrial Tachycardias are one of the rarer forms of supraventricular
tachycardia, accounting 
They tend to occur as a result of other cardiac or respiratory diseases.
They are characterised by a high heart rate ($\geq$ \unit{250}{bpm}) and
typically have evidence of an abnormal cardiac axis or P-wave morphology.
They are hard to treat with drugs, but radiofrequency ablation can be used with
a high probability of success.
Diagnosis of atrial tachycardia can be difficult due to a lack of data.
Attempts to locate the sites of the ectopic focus are current topics of clinical
research~\cite{kistler2006,Kahn2006,yamane2001}.
This study shows how the model can provide more data for clinicians to study.

\subsection{Model of Focal Atrial Tachycardia}

To model focal atrial tachycardia, the model developed in the previous chapter
is used.
Instead of pacing from the cells corresponding to the sinus node, various sites
around the atria are selected.
These sites are shown in Figure~\ref{fig:forward:sites}.
All the nodes which have active cells within 10 cells (\mm{3.3}) are excited via
direct current injection of \unit{2}{nS}\ for \ms{2}.
The resulting excitation wave is then allowed to propagate without interference
and the BSPM and ECG calculated.
The ECGs are classified by the P-wave morphology and the time of first
deflection in each lead.



