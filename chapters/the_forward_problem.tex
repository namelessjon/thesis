\chapter{Applications Of The Forward Problem}

The ECG is the first tool cardiac doctors turn to when diagnosis of a problem is
required.
ECG machines can be found in almost every hospital in the world.
A model of the atria and the surface potentials developed by the excitation of
the model can be used to guide diagnosis of a variety of conditions.
This can reduce the need for surgical procedures or suggest when they are
essential.

Whilst the inverse problem promises to reproduce the potentials on the heart
from the potentials on the surface, the technique has limitations.
For accurate solutions, patient specific geometries have to be constructed from
MRI scans.
There is also a need for complex lead systems, sometimes featuring more than two
hundred leads.
Also, many of the inverse techniques rely on `smooth' propagation patterns to
reduce the uncertainties in the technique which may not be found in pathological
cases.
A device which can perform such calculations automatically is a long way off,
both in terms of computational power required and complexities to resolve.

By contrast, diagnostic guides based on a forward solution can be of use to any
doctor.
They can also be used to further validate simulation studies of genetic or
diseased conditions, by comparison of the generated ECGs with those recorded
from real patients.
This chapter explores some of these predictions, using the model developed in
the previous chapter.

\section{Focal Atrial Tachycardia}

Atrial Tachycardias are one of the more uncommon forms of supraventricular
tachycardia.
They tend to occur as a result of other cardiac or respiratory diseases.
They are characterised by a high heart rate ($geq$ \unit{250}{bpm}) and
typically have evidence of an abnormal cardiac axis or P-wave morphology.
They are hard to treat with drugs, but radiofrequency ablation can be used with
a high probability of success.
Diagnosis of atrial tachycardia can be difficult due to a lack of data.
Attempts to locate the sites of the ectopic focus are current topics of clinical
research~\cite{kistler2006,Kahn2006,yamane2001}.
This study shows how the model can provide more data for clinicians to study.

\subsection{Model of Focal Atrial Tachycardia}

To model focal atrial tachycardia, the model developed in the previous chapter
is used.
Instead of pacing from the cells corresponding to the sinus node, various sites
around the atria are selected.
These sites are shown in Figure~\ref{fig:forward:sites}.
All the nodes which have active cells within 10 cells (\mm{3.3}) are excited via
direct current injection of \unit{2}{nS}\ for \ms{2}.
The resulting excitation wave is then allowed to propagate without interference
and the BSPM and ECG calculated.
The ECGs are classified by the P-wave morphology and the time of first
deflection in each lead.



