\chapter{Introduction}

Cardiac disease is one of the biggest causes of death in the UK, causing over
one third of all deaths.
In addition to the deaths, many more people suffer the after effects of a heart
attack or live with the difficulties caused by heart failure~\cite{bhf2008}.
These figures are duplicated across much of the developed world.

Mathematical modelling of the heart offers a way of gaining insight into the
cardiac processes and the mechanisms of cardiac disease.
It is a well established field of research with numerous international journals
and conferences discussing the findings.
Mathematical models allow physiological effects to be dissected and quantified
in ways that can be difficult for in vivo and in vitro experiments.
This can be used to inform both further experiments and clinical diagnosis and
treatment.

Mathematical models exist for many different types of cardiac cells and
cardiac tissue.
Many studies of cardiac tissue focus on the ventricles, with atrial studies less
common.
This is in part because failure of the ventricles can have a much more serious
impact, but in spite of this, the atria represent an interesting target to
study.
They have a complex electrophysiology and topology.

This study therefore aims to construct a series of models and tools for studying
the atria of the heart.
This involves modelling the atria and their processes on a wide variety of scales,
from the single cell to the whole atrium.
To enhance the clinical relevance of the study, a model of the atrium sitting
within the human torso will also be constructed.
This will allow the P-wave ECG to be calculated, the first view many cardiac
physicians will have of a failing atrium.
The tools will then be used to study a variety of factors that influence atrial
behaviour.

\section{Motivations}

\section{Aims}

\section{Synopsis}

This thesis consists of seven chapters.
In these chapters, a general background of the subject is given, before more
specific details of the toolkit developed are given.
There is then a section of experimental work with single cell, 1D, 2D and 3D
models.
A model of the body surface potential is then developed before being used in
experimental studies of the P-wave ECG.
Finally, conclusions and future work are given.

\paragraph{Chapter 1}: This chapter.
A introduction to the motivations and aims of the thesis and a summary of the
chapters contained within.

\paragraph{Chapter 2}: The physiological and mathematical background needed to
understand this thesis.
A description of the heart, with emphasis on the atria, is given, from the micro to the macro scale.
The normal functioning of the heart is described, both on cellular and whole
organ levels.
Mathematical models of cardiac tissue on all scales are introduced.
This includes a brief history of model development and information on both the
benefits and limitations of modelling.

\paragraph{Chapter 3}: The development and components of a cardiac simulation
toolkit.
A description of the technology and techniques which have gone into the
development of the cardiac simulation toolkit.
Details are given of the experimental protocols modelled by the toolkit.
The features offered by the toolkit are compared with the offerings of existing
toolkits.

\paragraph{Chapter 4}: Experimental studies in the atrium.
The atrial model developed in the thesis is presented, along with validating
information.
Experimental studies are presented, using the toolkit and simplified
versions of the whole atrium model.
These studies include a familial gene mutation, atrial fibrillation induced
remodelling and a novel current found in the human atrium.

\paragraph{Chapter 5}: The body surface potential or forward problem.
The mathematics of computing the body surface potential from the electrical
potentials in the heart are given along with implementation details of the
software used to solve them and the torso model used.
The effects of internal inhomogeneities in the torso on the generated ECG are
investigated.
The generated ECGs are compared with clinical data from both twelve lead and
body surface potential mapping.

\paragraph{Chapter 6}: Applications of the forward problem.
The body surface potential model is used in two clinical studies.
The first uses the model to validate an existing algorithm for predicting the
origin of focal tachycardia based on clinical data.
The second uses the model to investigate the causes of a novel clinical
phenomena, inverted P-waves at night.

\paragraph{Chapter 7}: Discussions and Conclusions
This includes a look to the future and the many avenues for future research
offered by the toolkit and models developed in the thesis.

