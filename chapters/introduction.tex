\chapter{Introduction}

Cardiac disease is one of the biggest causes of death in the UK, causing over
one third of all deaths.
In addition to the deaths, many more people suffer the after effects of a heart
attack or live with the difficulties caused by heart failure~\cite{bhf2008}.
These figures are duplicated across much of the developed world.

Mathematical modelling of the heart offers a way of gaining insight into the
cardiac processes and the mechanisms of cardiac disease.
It is a well established field of research with numerous international journals
and conferences discussing the findings.
Mathematical models allow physiological effects to be dissected and quantified
in ways that can be difficult for in vivo and in vitro experiments.
This can be used to inform both further experiments and clinical diagnosis and
treatment.

Mathematical models exist for many different types of cardiac cells and
cardiac tissue.
Many studies of cardiac tissue focus on the ventricles, with atrial studies less
common.
This is in part because failure of the ventricles can have a much more serious
impact, but in spite of this, the atria represent an interesting target to
study.
They have a complex electrophysiology and topology.

This study therefore aims to construct a series of models and tools for studying
the atriums of the heart.
This involves modelling the atria and their processes on a wide variety of scales,
from the single cell to the whole atrium.
To enhance the clinical relevance of the study, a model of the atrium sitting
within the human torso will also be constructed.
This will allow the P-wave ECG to be calculated, the first view many cardiac
physicians will have of a failing atrium.
The tools will then be used to study a variety of factors which influence atrial
behaviour.


\section{The Heart}

The heart's role is to pump blood around the body, driving in the circulation of
the blood and everything contained within it.  It is one of the most important
organs in the body and any malfunction in its behaviour could be fatal in very
short order.  It begins beating in the early stages of pregnancy and continues
until death, hopefully many decades later.  It beats at an average rate of
around 70 beats per minute (bpm) for the adult male and 75 bpm for the adult
female.

The heart is not, as popular belief would have it, the seat of human emotion.
The functioning of the heart is modulated by such emotion however, slowing when
we are calm and increasing in rate quite dramatically when we are excited or
afraid.  Despite being influenced by the brain and our emotional states, the
heart drives itself, rather than having the pace-making initiated outside the
organ.

\subsection{Location of the Heart}

The human heart sits in the centre of the chest, the bulk of it extending the
left-hand side of the chest cavity, inside a fibrous sac called the pericardium.

\subsection{Structure of the Heart}

The structure is mostly muscle, anchored to a collagenous `skeleton', known as
the annulus fibrosus located at the atrio-ventricular junction.  This muscle is
different from the `smooth' skeletal muscle, in both structure and behaviour.

\subsubsection{The Four Chambers}

The heart has four chambers, two atria and two ventricles.  The atria receive
the blood from the circulatory system and force it into the two ventricles,
which then contract and force this blood out and around the lungs and body.
These chambers are known as the left and right atria and the left and right
ventricles.  The left hand side of the heart in humans is much more developed
than the right.  This is due to their differing roles in the circulation of the
blood.

The right and left atria are smaller than their respective ventricles and have
much thinner walls, because they need to develop much less pressure.  The right
atrium receives the blood from the circulatory system which is de-oxygenated,
and passes it on to the right ventricle.  The left atrium receives the highly
oxygenated blood from the lungs, and passes it onto the left ventricle.  The two
atria are separated by a thin muscle wall known as the intra-atrial septum.
This prevents the mixing of blood between the two atrial chambers.

The differences between the right and left ventricles are much more pronounced
than those between the right and left atria.  The right ventricle must merely
pump blood around the lungs and developing too high a pressure there could
actually damage the delicate structures.  By contrast the left ventricle must
develop enough pressure to drive blood around the whole body and as such it is
much more muscled.  Again, the two ventricles are divided by the ventricular
septum.

\subsubsection{The Fibrous Structure}

The fibrous structure of the heart is perhaps as important to its correct
function as the muscular structure.  The annulus fibrosus, or central fibrous
body, sits between the atria and the ventricles and provides an anchor for the
muscle's contraction.  It is formed from connective tissue and it electrically
isolates the atria from the ventricles.  The valves which separate atrium from
ventricle and ventricle from artery are also part of this structure.

\section{Mathematical Models of the Heart}

Cardiac tissue has been modelled mathematically for about fifty years.
While initial models were simple, there are now models of high sophistication
available.
These models are capable of reproducing both healthy and pathological behaviour
with some accuracy.

\subsection{Categorising Myocyte Models}

Cellular models tend to be classified in two ways.
The first differentiator is the level of detail employed.
The second is on which cellular processes are modelled.

Biophysically detailed models are complex.
They consider the interactions of several different currents, and potentially
intra- and extracellular ion concentrations, reservoirs as well as other
details.
The second type are simplified, phenomenological, models.
These do not consider individual ion concentrations but instead just reproduce
one desired factor, typically the action potential profile.

Models, whether biophysically detailed or not, can concern themselves with the
cellular electrophysiology, the mechanical contrations or both.
This thesis concerns itself just with electrophysiological models.
Models of the mechanics are not considered and are not treated in this
description of mathematical modelling.

\subsection{A Brief History Of Cardiac Myocyte Modeling}

The first model of cardiac electrophysiology was published by
Noble~\cite{Noble1962}, and modelled Purkinje fibre (A specialised part of the
ventricular conduction system) action potentials.
Shortly thereafter, various refinements were published, as further experimental
data became available.
This was eventually followed by the Beeler--Reuter~\cite{Beeler1977}\ model of
the guinea pig ventricular myocyte.

The Luo--Rudy guinea pig ventricular myocyte model was first published in
1991~\cite{Luo1991} and was then sub-sequentially majorly revised and
republished in 1994~\cite{Luo1994}.
The first Luo-Rudy model was based on the Beeler--Reuter model, though with updated channel behaviours
and more complex potassium channels.
The second Luo-Rudy model was the first of the `second generation' models,
with fluctuations in all ion concentrations, and a much more detailed series of
equations for calcium handling.

In 1998, the Courtemanche--Ramirez--Natel\cite{CRN1998}\ (CRN) model and the
Nygren~\cite{Nygren1998}\ model were published.
These were both models of the human atrium.
They were also both second generation models, with detailed calcium handling.
Also in 1998 Fenton and Karma published their phenomenological
Fenton--Karma~\cite{Fenton1998}\ model, which used just three channels to
reproduce the shape of the ventricular action potential.

\subsection{Mathematical Models of Myocytes}

Mathematical models of cardiac myocytes are built from a few simple assumptions
and considerations.
These concern the behaviour of the cellular membrane and ion channels.
The important ones are detailed here.

\subsubsection{Electric Circuit Model}

The electrical circuit model of the cell membrane underpins much of the work in
modelling the behaviour of cardiac myocytes.
It is however a very simple concept.
Since the membrane seperates charges, it may be considered as a capacitor, as
shown in figure~\ref{fig:intro:math:circuit}.
Since there is no net buildup of
charge on either side of the membrane, any ionic current, \ii{ion}, must be
countered by a capacitive current, and so
\begin{equation}
C_{m}\frac{dV_{m}}{dt} + \ii{ion} = 0
\label{eqn:intro:math:basic}
\end{equation}
where $V_{m}$ is the membrane voltage, and is defined as the difference between
the internal potential, $u_{i}$ and the external potential, $u_{e}$ or
\begin{equation}
V_{m} = u_{i} - u_{e}
\label{eqn:intro:math:vm}
\end{equation}
When multiple currents are considered, the total inward and outward currents are
summed.
The difficulty comes in determining the form of \ii{ion}, which varies widely
depending on the nature of the channel or pump in question.

\subsubsection{The Nernst Equilibrium Potential}

The Nernst equation is an important equation in electrophysiology.
It describes how the difference in ion concentrations on two sides of a
semi-permeable barrier can result in a potential difference across the barrier.
Any voltage dependant factor of a current of an ion, S, includes a reversal
potential, V$_{S}$, equal to the Nernst potential.
At the reversal potential, the current falls to 0.
When equilibrium is reached the potential difference, $V_{S}$, across the
membrane is given by
\begin{equation}
V_{S} = \frac{RT}{zF}ln\left( \frac{[S]_{e}}{[S]_{i}} \right) 
\end{equation} 
where subscripts $i$ and $e$ denote internal and external concentrations of S,
$R$ is the universal gas constant, $T$ is the absolute temperature, $F$ is
Faraday's constant and $z$ the charge of the ion $S$.

The Nerst Potential applies only to a single ion concentration.
This is not as large a limitation as it first might seem.
Whilst the action potential involves many ion species, many channels are ion
specific and thus the Nerst potential applies across them.

\subsubsection{The Hodgkin-Huxley Equations}

Many mathematical models of cardiac myocytes feature one or more
`Hodgkin--Huxley' channels.
Hodgkin and Huxley developed them in a now classic series of papers concerning
the current flow through the membrane of the squid giant axon.
They characterise the current flow with elegance and surprising accuracy.
It is important to note that the Hodgkin--Huxley equations consider the bulk
behaviour of the many thousands of individual channel structures distributed
across the membrane, not the behaviour of one single channel.

Hodgkin and Huxley started with a very simple assumption.
The current flow through a channel on the membrane, $I_{S}$ is given by
\begin{equation}
I_{S} = g_{S}\left(V-V_{S}\right)
\label{eqn:intro:math:hh1}
\end{equation}
where $g_{S}$ is the channel conductance, $V$ the membrane potential and $V_{S}$
the Nernst potential for the ion S.
Equation (\ref{eqn:intro:math:hh1}) assumes that the channel is selective for
one ion species S, and that the current is a simple linear function of the
voltage across the membrane.

With this underlying assumption, Hodgkin and Huxley set out to accurately map
the behaviour of the current with regards time and voltage.
The following section explains the description for the sodium current, \ii{Na}.
From the form of the sodium channel under voltage clamp conditions, it is
reasonable to expect $g_{Na}$ obeys a differential equation of the form
\begin{equation}
\frac{dg_{Na}}{dt} = f\left(v,t\right)
\label{eqn:intro:math:hh2}
\end{equation}
where $v=V-V_{Na}$.
However, the form of $g_{Na}$ is complex.
While remaining at the same voltage, the conductance at first increases and then
tails off.
It appeared that there were two processes at work, one that turned the current
on, and one that turned it off.
Hodgkin and Huxley realised that it would be easier to write $g_{Na}$ as a
function of two different variables.
One which corresponded to the turning on and one to the turning off of the
channel.
This leads there being an activation variable, called $m$, an inactivation
variable, called $h$ and that the current would be some linear combination of the
two, multiplied by a constant conductance factor $\bar{g}_{Na}$.
The two variables $m$ and $h$ would both satisfy a differential equation such as
\begin{equation}
\frac{dm}{dt}=\alpha_{m}\left(v\right)\left(1-m\right) -
\beta_{m}\left(v\right)m
\label{eqn:intro:math:dmdt}
\end{equation}
where $\alpha_{m}$ and $\beta_{m}$ are functions of $v\,\left(=V-V_{Na}\right)$.
As an activation $m$, $\alpha_{m}$ and $\beta_{m}$ are such that $m$ is
initially small but increases with the potential.
As $h$ is an inactivation, $\alpha_{h}$ and $\beta_{h}$ give an initially high
value of $h$ that then decays, inactivating the channel.

The form proposed for $g_{Na}$ by Hodgkin and Huxley was
\begin{equation}
g_{Na}=\bar{g}_{Na}m^{3}h
\label{eqn:intro:math:gna}
\end{equation}
where all symbols are as defined previously.
The decision to raise $m$ to the third power was based on the rate of increase
observed in voltage clamp experiments.
It is interesting to note that when the structure of \ii{Na}\ was examined in
detail, it was discovered that the channel has three structures which open to
allow current to flow.
A second structure, the `ball and chain' then acts to close the channel.

Many different channels are modelled as Hodgkin--Huxley channels.
Different channels have different activation and inactivation variables.
These variables are modulated by different $\alpha$ and $\beta$ equations.

\subsubsection{Markov Chain Currents}

\subsection{Selected Myocyte Models}

There are tens, perhaps hundreds, of myocyte models of varying complexity and
accuracy.
There are relatively few models of atrial myocytes in the human however.
A brief description of the foremost two are given here, as well as a description
of one of the most adaptable phenomological models.

\subsubsection{The Courtemanche--Rameriz--Nattel Model}

\subsubsection{The Nygren Model}

\subsubsection{The Fenton--Karma Model}

\subsection{Action Potential Propagation}

Single myocyte models are important, and can tell us much about the heart in
disease and health.
The heart is not made up of isolated myocytes however.
Whilst current computational power does not allow myocytes to be modelled on an
individual cellular basis for the whole heart, continuum models of propagation
have been developed.
These are summed up in the bidomain equations, and their simplification, the
monodomain equations.

\subsubsection{The Bidomain Equation}

The bidomain equation comes out of basic electromagnetic theory and several
assumptions about the nature of cardiac tissue~\cite{Tung,STUFF}.
\begin{enumerate}
    \item The cardiac tissue contains two continuous, simply connected domains, the intracellular and extracellular domains.
    There is no detailed consideration of the fine points of geometry.
    \item The intra- and extracellular domains overlap and fill all of the cardiac muscle. Each point lies in both domains.
    \item Charge does not accumulate.
\end{enumerate}

The derivation is not that difficult to work through, and it results in the bidomain equations
\begin{align}
\underline{\nabla}\cdot\left(\left(M_{i}+M_{e})\right)\underline{\nabla}u_{e}\right) + \underline{\nabla}\cdot\left( M_{i}\underline{\nabla}V_{m}\right) &=& 0
\label{eqn:intro:math:bidom1}\\
\underline{\nabla}\cdot\left(M_{i}\underline{\nabla}V_{m}\right) + \underline{\nabla}\cdot\left(M_{i}\underline{\nabla}u_{e}\right) &=& \chi C_{m}\frac{dV_{m}}{dt} + \chi{I_{ion}}
\label{eqn:intro:math:bidom2}
\end{align}
where the suscript $i$ and $e$ refer to the intra- and extracellular quantities, $M$ is a matrix of conductivities, $\chi$ represents the surface-to-volume ratio of the cells and all other quantities are as defined previously.
The bidomain equations are a coupled set of a parabolic and elliptic differential equation.

Boundary conditions for the bidomain equations vary, though the most common ones are described here.
First, no intracellular fluxes leave the heart.
Second, the body is assumed to be a passive conductor that is isolated at the outer surface.
The body potential at the surface of the heart is the extra-cellular potential at the surface of the heart.

\subsubsection{The Monodomain Equation}

While the bidomain equations represent a good tool for modelling some of the complexities of cardiac conduction, they are very demanding to solve, necessitating finding the solution to coupled parabolic and elliptic differential equation sets.
The monodomain equation is the result of one simplifying assumption made to the bidomain equations.
For the monodomain equation, we assume that the anisotropy ratio, $\lambda$, is the same for the intra- and extra-cellular fluids at all points.
\begin{equation}
M_{i} = {\lambda}M_{e}
\label{eqn:intro:math:mratio}
\end{equation}
This assumption is not a very physiological one, but the simplification it
allows is significant and so it is quite commonly used.

Substituting (\ref{eqn:intro:math:mratio}) into (\ref{eqn:intro:math:bidom1})
and (\ref{eqn:intro:math:bidom2}) and rearranging reduces the pair of equations
to one single equation for the membrane potential
\begin{equation}
\frac{\lambda}{1+\lambda}\underline{\nabla}\cdot\left(M_{i}\underline{\nabla}V_{m}\right) = \chi C_{m}\frac{dV_{m}}{dt} + \chi{I_{ion}}
\label{eqn:intro:math:mono}
\end{equation}
Typically, the factor of ${\lambda}/\left({1+\lambda}\right)$ is folded into $M_{i}$ to give the diffusion tensor $D$.
In 1D, this is the cable equation, which is very widely used.
The values of the components of the tensors $M_{x}$ or $D$ may be determined experimentally, or from a comparison of conduction in real and virtual tissue samples.

\section{The Electrocardiograph}

The electrocardiograph, or ECG, was developed by Einthoven and colleagues at the
turn of the 20th century.
The Einthoven ECG used the string galvanometer, developed by Einthoven, to
record the potential differences between three sets of electrodes, or leads.
These electrodes were placed on each arm and on the left leg.
These three electrodes form the bases of many ECGs recorded to this day.
The string galvanometer was highly sensitive electrical recording device for its
time, and was developed by Einthoven.
For the development of the ECG and the string galvanometer, Einthoven was awarded
the nobel prize in 1924~\cite{Kligfield2002,TEXTBOOKS}.

Since Einthoven's day, the ECG has been continually refined.
This refinement has included both improvements in the way that individual leads
are measured, and new leads that are recorded.
In addition, there have been a number of specialised lead sets developed for
particular purposes, such as exercise recording and 24 hour measurement.
Research into new lead sets, to better understand the functioning of the heart,
continues to this day~\cite{PAPERS}.

\subsection{Lead Theory}

In Einthoven's original conception of the heart and the electrical field it
produced, he developed the ``Einthoven Triangle''.
The Einthoven triangle is an equilateral triangle.
At its corners sit the three electrodes and the sides of the triangles are the
leads themselves.
At the centroid of the triangle sits the heart, which is represented by a
single, stationary, time varying dipole.
The potentials measured at the three electrodes are the potentials assuming the
system is two dimensional, homogeneous and infinite in extent.
This was not entirely incorrect.

Modern lead theory was developed in the 1950's.
In a series of three papers McFee and
Johnston~\cite{McFee1953,McFee1954a,McFee1954b}\ set out their concept of the
``lead field'' which is still considered applicable today.
This theory, which was a further generalisation of work by Burger and van
Milaan, allowed for a heart which consisted of distributed dipole sources
sitting in a three dimensional, finite and inhomogeneous medium.



\section{The Forward Problem}

The so-called `forward problem' seeks to find a relationship between the
electrical activity in heart and the potentials observed outside the heart,
most usefully on the surface of the body.
When looked at another way, it is a method of finding the lead field, $\vec{L}$,
for a given body.
The problem has been approached in a number of ways; experimentally, clinically
and numerically.

\subsection{Uses of the Forward Problem}

The original investigations into the forward problem very much concerned
themselves with finding $\vec{L}$.
They sought to refine the understanding of such concepts as the einthoven
triangle and the precordial electrodes and to more closely relate the differing
potentials observed with what was happening in the heart.
The experiments in the forward problem led to the lead field theory of McFee and
Johnson~\cite{McFee1953}\ and the Frank vectorcardiogram~\cite{Frank1956}.

Modern investigators of the forward problem use the solutions to perform similar
investigations, such as the development of new lead systems~\cite{Ihara2007}.
Forward solutions are also used for investigations into ischeamia, ventricular
and atrial fibrillation, conduction defects, amongst many.
Forward solutions also form the basis, or are used in the refinement, of inverse
solutions.

\subsection{A Brief History of the Forward Problem}

As has been mentioned, the initial uses of the forward problem were to determine
how the activity of the heart related to the leads.
These initial models tended to be real and physical models.
Towards this end, one of the first models constructed was by Burger and van
Milaan~\cite{Burger1946}, who constructed a one third life-size model of the
torso, filled with electrolyte solution.
The model had a cork spine and sandbags to represent the lungs.
The measurements from this model formed the basis for their refinements to the
lead theory.

Of the early mathematical models Brody~\cite{Brody1956}\ is one of the most
significant.
His was a model of the influence of the highly conductive blood masses of the
heart on the measured electrical field.
The `Brody Effect' is still used in literature today.

In the 1960s, numerical models of the torso started to
appear~\cite{Barr1966,Barnard1967}.
These initial models generally had a handful of dipoles, or even just one, to
represent the heart.
These were used to investigate the influence of dipole position and
heterogeneities on the body surface potential.

In 1971 Rush~\cite{Rush1971} published his torso model.
This was probably the ultimate torso model, literally and figuratively.
The model was twice lifesize and incorporated internal inhomogeneities
representing lungs, heart, heart blood, great blood vessels, liver, fat,
skeleton and anisotropic skeletal muscle--though the model could be used without
them as well.
The cardiac activity was represented by a number of dipole sources which were
located in the heart region of the torso.

In the 1980s, models of the heart and the torso gained more
sophistication~\cite{Gulrajani1983,Horacek1987}, though they were still generally
ventricular models, not models of the whole heart.
Models of this era routinely used 10s of dipole sources.
These models still did not include even primitive electrophysiology models,
although towards the end of the decade, propagation of the excitation wavefront
was being modeled~\cite{Aoki1987}.

Moving to the nineties and the turn of the millenium, the availability of
medical imaging tools such as CT and MRI scans allowed more accurate models to
be constructed~\cite{Weixue1993,Weixue1996}.
This is also when the first models began to appear which incorporated the
atrium~\cite{Weixue1996,vanDam2005}.
Many of the models published in this era use myocyte models to simulate the
propagation, some even using biophysically detailed models.

\subsection{Numerical Approaches}

Numerical solutions to the forward problem involve solving Maxwell's equations
within the torso.
The same assumptions which were made for the lead field theory
($\S$\ref{sec:intro:ecg:lead_theory}) are valid here.
To reiterate them briefly, they state that the body is a linear volume
conductor.
There are no inductive or capacitive effects.
Furthermore, due to the finite (and small) size of the body, there are no
propagation effects to consider.
The field in the body, $\vec{E}$ is therefore given by
\begin{equation}
\label{eqn:intro:forward:maxwell}
\vec{E} = -\nabla\phi
\end{equation}
where $\phi$ is a scalar potential field.
Current flow in the torso under these conditions is given by Ohms law, which
states the total current flow, $\vec{J}$, is
\begin{equation}
\label{eqn:intro:forward:ohm}
\vec{J} = \mathbf{\sigma}\vec{E} + \vec{J^i}
\end{equation}
where $\mathbf{\sigma}$ is the conductivity tensor and $\vec{J^i}$ is an
impressed, or applied, current density which is generated by active sources,
i.e. the heart.

Since the total current flow is solenoidal, i.e. the net current flow into and
out of a volume is zero, via (\ref{eqn:intro:forward:ohm}) we have that
\begin{equation}
\label{eqn:intro:forward:ohm2}
\nabla\cdot\vec{J} = 0 = \nabla\cdot\left(\mathbf{\sigma}\vec{E} + \vec{J^i} \right)
\end{equation}
This can alternatively be written as
\begin{equation}
\label{eqn:intro:forward:poisson}
\nabla\cdot\vec{J^i} = \mathbf{\sigma}\nabla^2\phi
\end{equation}
If the body is homogeneous and isotropic, it becomes
\begin{equation}
\label{eqn:intro:forward:poisson2}
\nabla^2\phi = \frac{\nabla\cdot\vec{J^i}}{\sigma}
\end{equation}
which is Poisson's equation.

Finding $\phi$ for a given $\vec{J^i}$ can be achieved using a volume or
boundary based approach.
Volume based approaches (\cite{Seger2004,Klepfer1997,Keller2007}) involve dividing the body
up into volumes of different conductivity.
The solution for $\phi$ is found using a finite element or finite difference
method.
Boundary element based approaches
(\cite{Barr1966,Clayton2002,Gulranjani1989,Weixue1996}) divide the torso into
regions of isotropic and uniform conductivity.
The potential, $\phi$, is found on the surfaces of these regions.

Volume based methods allow for a more complex model.
Interior volumes can have internally varying or even anisotropic conductivity.
They tend to be much more computationally intensive to solve, however, because
the problem space is still three dimensional.
In addition, the required element size can be quite small, especially for finite
difference approximations.
Keller~\cite{Keller2007}\ used a model with more than ten million torso
elements, for example, whilst the finite element model used by
Klepfer~\cite{Klepfer1997}\ has approximately one million elements and 168,000
nodes.

Boundary element based methods by contrast tend to be much simpler to solve.
Typical model sizes are of the order of a thousand to ten thousand elements, as
they reduce the problem space to a series of two dimensional surfaces.
This reduces the computational effort required.
However, since the solution is only computed at the surfaces, regions can only
be uniform and isotropic.






\section{Cardiac Simulation Toolkit}

To extract useful results from cardiac modelling two principle components are
needed, as has been previously mentioned: the mathematical models used for
cardiac activity and a computational representation of the experimental
protocols followed to produce results.  A cardiac simulation toolkit provides
one or both of these components to an investigator, removing the need for the
investigator to implement those components themselves.  The word toolkit is
chosen to differentiate from the relatively more common practice of just
providing individual cell models.  A toolkit implies more than one model, or
variants on a model, with a common calling interface and implementation of one
or more experimental protocols or a non-programmatic way of setting up such
protocols---whether through a GUI or through command-line options or
configuration files.

Toolkits have a number of advantages to the investigator, including a reduction
in the work required to bring a paper to publication, increased consistency both
within and outside of the group, fewer implementation mistakes, increased
possibilities for collaboration and verification and a greater accessibility to
cardiac simulations for non-programmers.  The disadvantages of toolkits are not
as numerous, but they too will be considered.  A toolkit will need auditing, to
ensure that the results it gives are accurate.  Toolkits can also constrain the
avenues of experimentation pursued by an investigator by making protocols easier
or harder to implement.  Finally, the additional complexity makes adding any new
feature a little harder---there is a `feature cost'.

\subsection{Advantages of Cardiac Simulation Toolkits}



Cardiac simulation toolkits have existed in one form or another for some
time and include both commercial and open source 
Several cardiac simulation systems have previously been released, one of
the first of which is the OXSOFT HEART~\cite{Noble-1999}.

