\chapter{Conclusions}


This was an investigation into models of cardiac tissue, with a special focus on
atrial tissue\cite{foo}.
It included whole atrial models, which were then extended beyond the heart to
simulate the P-wave ECG.
A toolkit providing access to efficient implementations of various experimental
protocols was developed.
A model of the atrium based on biophysically detailed myocyte models, including tissue heterogeneity and anisotropic
conduction, was constructed.
The atrial model was used as the basis of a model of the P-wave ECG which was
solved using a boundary elemental formulation.
All of the tools and models developed were used to perform physiological
studies of tissues in healthy and diseased states.

The cardiac toolkit which was developed provides easy access to a wide variety
of experimental protocols.
These protocols are used to assess the physiological impact of a gene mutation,
drug action, hormonal effect or other electrophysiological modification.
They include protocols which act on single cells and also on one dimensional
idealisations of cardiac tissue.

The implementations of these protocols focused on efficient algorithms
which take advantage of the computational nature of the models.
It was important that this did not compromise on the physiological accuracy and
level of detail employed.
This was accomplished in part through the use of lookup tables which reduced the
computational effort needed to solve \ms{1}\ of cardiac activity compared to an
implementation without such tables without impacting measured physiological
characteristics significantly.
Such a saving might be considered a `cell level' performance optimisation.
Greater savings can be observed through the use of `protocol level' performance
optimisation.
These protocol level optimisations are where the computational nature of the
models can really be exploited.
Optimisations include storing of cellular state after the pre-pacing part of the
protocol and adaptive stepping when tracking response curves of varying slope.
There was also the novel use of a basic computer science algorithm, the binary
search, to determine the limits in a number of experimental protocols.
This can be used, with a sensible choice of initial guesses, to reduce the total
number of cases which must be tested by an order of magnitude.

The toolkit also offers an easy way of specifying and running 2D simulations.
Simulation, even of irregular geometry, is simplified.
Utilities can be used to convert quantified colour bitmaps into simulation
geometries with heterogeneous electrophysiology.
A variety of stimulation protocols can be specified, including both current and
voltage stimuli.
The simulations are accompanied by regular outputting of gif snapshots of the
voltage, to allow simulations to be assessed as they run.
In addition, the tissue simulation code is parallelized using a shared memory
paradigm with OpenMP.


