\chapter{Constructing a Cardiac Simulation Toolkit}

\section{Simulation Environment}

The simulation environment provided by the cardiac toolkit is intended to be as
portable as possible, so that numerical experiments may be run on whichever
platforms are appropriate.  To this end, all the data input structures are based
on open standards, or simple binary formats.  The output formats provided by the
various driver programs are also in simple binary or ASCII formats, to allow
them to be easily visualized with both commercial and open source visualisation
tools.  The results presented later in this chapter were performed on desktop
computers with a XX GHz Althon X2 chip and 1 GB RAM and on Horace, the local HPC
facility.  Horace has 24 compute nodes, each one consisting of four Intel Itanium2
Montecito Dual Core 1.6GHz processors, 16GB RAM and up to 512GB of local scratch
space.  The nodes are connected by a high speed Quadrics QsNetII
interconnect~\cite{horace}.  Horace provides compilers for both Fortran and C,
and for both the MPI and OpenMP parallelization libraries.

\subsection{Implementation}

The experimental protocol drivers and the cellular models were implemented in
the C programming language.
Much of the supporting code and supplementary tools was implemented in the ruby
programming language.
The cellular models currently implemented are based on the Hodgkin-Huxley
formalism, although there is no fundamental reason why a Markov chain based
model could not be included.
Inter-cellular coupling for propagation of excitation over a strand or tissue
was implemented using the monodomain equations.

\subsubsection{Cellular Models}

The cellular model used for much of the developmental process was the
Courtemanche et al. human atrial myocyte model~\cite{CRN98}.
Also currently implemented are the Nygren et al. human atrial myocyte
model~\cite{nygren98} and the four variable formulation of the Fenton-Karma
minimal variable model~\cite{Bueno-Orovio2008}.
These cellular models describe the behaviour of a cell using coupled systems of
non-linear ordinary differential equations.
The ODEs represent the concentrations of intra- and extra-cellular ion species
and the flow of current through ionic channels in the cell membrane or between
intra-cellular compartments, or their notional equivalents in the case of
minimal variable models.

These equations were generally solved using the simplest time-stepping method
available, the explicit Euler method.
To improve performance and stability, some variables were integrated using the
Rush-Larsen method.
More complex integration schemes were tried, but did not significantly improve
performance to compensate for the greater complexity.

\subsubsection{Monodomain Equations}

The monodomain equations were used to couple multiple cells together to describe
a tissue over which excitation could be conducted.
The rate of change of membrane potential, $V$, is given by
\begin{equation}
\label{eqn:toolkit:monodomain}
\frac{\partial V}{\partial t} = D \nabla^{2} V - \frac{\ii{ion}}{C_{m}}
\end{equation}
where $D$ is a constant representing the diffusivity of transmebrane through
space, \ii{ion}\ represents the total trans-membrane of a cellular model, such
as \ii{tot}\ from (\ref{eqn:intro:crn}) and $C_{m}$ is the membrane capacitance
of the cell.
A finite differences approach is used to discretize the model in 1D or 2D with
an explicit Euler scheme used to advance the timestep.
A discussion of the merits of the monodomain and bidomain equations can be found
in the introduction.

\subsubsection{Strand Model}

\begin{figure}[h]
\setlength{\unitlength}{1mm}
\begin{picture}(130,30)

\multiput(0,10)(20, 0){7}{\usebox{\cell}}
\multiput(10,14)(20, 0){6}{\usebox{\resistor}}
\end{picture}
\caption[Schematic diagram of a 1D strand]{\label{fig:toolkit:strand}
Schematic diagram of a 1D strand.
The strand is 7 nodes long, with each node represented by the blue square.
The electrical activity at each node is represented by a mathematical model of
the trans-membrane currents.
The cell is coupled to its neighbours through resistances, the black rectangles.
}
\end{figure}

The 1D strand model is used for several experimental protocols, as
a computationally cheaper alternative to a full tissue model.
The 1D strand model consists of a number of nodes, typically 200 or 300, which
are coupled electrically at the ends of the cells as shown in
Figure~\ref{fig:toolkit:strand}.
The electrical activity at each node is modelled via a cellular
electrophysiology model.
Electrical conduction between the nodes is handled via a 1D formulation of the
monodomain equations using a 3-node approximation of the Laplacian, with no
flux boundary conditions.



\subsubsection{Sheet Model}

\begin{figure}[h]
\setlength{\unitlength}{1mm}
\begin{picture}(130,130)
\multiput(0,0)(20, 0){7}{
    \multiput(0,0)(0,20){7}{
        \usebox{\cell}
    }
}
\multiput(10,4)(20, 0){6}{
    \multiput(0,0)(0, 20){7}{
        \usebox{\resistor}
    }
}
\multiput(4, 10)(20, 0){7}{
    \multiput(0, 0)(0, 20){6}{
        \usebox{\vresistor}
    }
}
\end{picture}
\caption[Schematic diagram of a 2D sheet]{\label{fig:toolkit:sheet}
Schematic diagram of a 2D strand.
The strand is $7\times7$ nodes in size, with each node represented by a blue
square.
The electrical activity at each node is represented by a mathematical model of
the trans-membrane currents.
The cell is coupled to its neighbours along the cardinal directions through
resistances, the smaller black rectangles.
}
\end{figure}

The 2D sheet model is used for several experimental protocols, as well as more
general numerical experimentation.  The 2D sheet model consists of a grid of
nodes, coupled electrically along the cardinal directions of the grid.  The
electrical activity at each node is modelled via a cellular electrophysiology
model.  Conduction of the electrical excitation between the nodes uses the
monodomain equations with no flux boundary conditions applied at all tissue
boundaries and a 5-node approximation of the Laplacian.  The square sheet model,
used in several of the numerical experimental protocols described later, is
typically $375\times375$ nodes, representing 140,625 cellular models.  Two dimensional
idealizations of physiological preparations can have many more nodes, to on the
order of $10^{6}$ cellular models.  These idealizations can often be quite
irregular and so to allow easy and effective partition of workload across
multiple processors, the tissue map is decomposed into a 1D array which contains
references to the neighbouring cells.


\subsection{Parallelization}

Some parts of the toolkit require the modelling of large numbers of cells, on
the order of tens or even hundreds of thousands of cellular models in two
dimensional sheets.  Solving all the equations involved takes a significant
amount of time and so it is desirable for such simulations to be parallelized so
that the work involved can be split over several processors.  This can have
advantages beyond merely having eight rather than one cores worth of
computational cycles working on solving the equations.  Splitting the work over
multiple cores can also increase the amount of cache available, allowing for
more efficient operation of the solvers.
There are two libraries widely available for parallelization,
OpenMP~\cite{OpenMP}\ and the Message Passing Interface~\cite{MPI} (MPI).
They are based around different paradigms for parallelism.

The OpenMP library is based around the `shared memory' paradigm.
Under the shared memory paradigm, there is only one process and this has access
to all of the memory used in the program.
The parallelism is achieved through the use of threads which divide up the work
between themselves.
This makes the implementation quite simple, but also limits the maximum number
of computer cores which can be assigned to work on any individual execution of
the problem.

MPI is based around the `message passing' paradigm.
The message passing paradigm involves multiple separate processes which use
communication via the messages they pass between themselves to work.
Each process has its own memory, and the only way for information in one process
to reach another process is by explicitly passing it through messages.
Because the processes each have their own memory, there is no requirement that
the processes are on the same computer.
This allows (theoretically) any number of cores to be applied to solving the
problem.
However, the explicit nature of message passing can make implementing a program
harder and in addition, there can be significant lag due to the physical
separation of the computers which can reduce efficiency.

For the parallelism involved in the library, the OpenMP parallelism library was
chosen.
This choice was made for a number of reasons.
Implementations of OpenMP can be found on many systems, with gcc, icc and the
Sun compiler all having an OpenMP implementation.
This makes the library suitable for use on both HPC systems and on modern
multiple core desktops.
In addition, for those systems on which running in parallel is not desirable,
the same code can be compiled serially.
Whilst an MPI implementation would allow more processors to be used to execute a
task, in general 8 cores of horace were found to be sufficient for running most
jobs and in compiling for MPI some of the flexibility would be lost from the
library.

\subsection{Performance Optimisation}

Optimisation is the process of improving a given quantity.
In the context of performance, this involves reducing the running time whilst
preserving the accuracy of the solution.
The benefits of this should be obvious.
Code which runs faster lets results be gathered sooner or allows more cases to
be considered.
Optimisation can be of particular benefit to a library, which is intended to
facilitate code re-use; the benefit of a little work can be garnered many
times.

In general, all performance optimisation techniques involve reducing the number
of operations required to compute the final result.
This can take a number of forms.
The simplest one is the choice of the compiler and the compile flags used, both
of which can have a significant influence on the total computation time.
However, moving beyond the compiler, a choice of algorithm can also be important
in reducing the time taken.
Several of the techniques used are presented here.

\subsubsection{The Compiler}

The library has been compiled using the GNU C compiler (gcc)~\cite{gcc}, the
Intel C compiler (icc)~\cite{icc} and the Sun microsystems compiler.
All three have OpenMP implementations available and all three are capable of
performing a number of optimisations, controlled via flags.
The most important
aspect of the
optimisations is that they should not alter the behaviour of the floating point
handling, as this could have significant impact on the final result computed.
Despite this caveat, the results of applying certain optimisation flags can be
quite significant.

\subsubsection{Caching of Computed Values}

Moving beyond the compiler, one of the simplest forms of optimisation is to only
calculate each value once, if at all possible.  This can be done in a number of
ways and the library developed here implements two such methods for saving
computational time.

State saving is one of the most direct ways of caching computed values.  At a
particular point in the simulation, all of the state variables of the
system are copied into an intermediate location.  This might be a file on disk or
to another location in memory.  If the state is written out to a file, that file can
be used as a `save point', allowing the simulation to be continued from that
point in the future, ensuring work is not wasted.

When copied to another memory location, this allows the program to return to
that point in the future.  This is useful in modelling many experimental
protocols, which often call for a number of `conditioning' pulses to allow the
cell or model to settle.  The state can be saved after the conditioning pulses
and then the actual tests can be performed quickly, saving the execution of
several seconds of simulated activity.  This technique should obviously only be
used for cells in the Hodgkin-Huxley formalism which are deterministic and thus
give identical results whether the state is saved or not.  Using such a
technique with a cell that has a number of stochastic components could
potentially affect the quality of the results.

The second way in which caching can be employed is in the creation of `lookup
tables'.
A lookup table is a pre-computed table of the values an expression can
take.
When the expression would normally be evaluated, the table is used
instead, replacing what might be a complicated expression with a single array
lookup.

To efficiently pre-compute values for a lookup table, two things must be true.
The tabulated expression must depend on only one variable.
The tabulated expression must also be sufficiently `complex' or time consuming
to compute.
The requirement for complexity is perhaps the most obvious one.
The most time that a lookup table can save is the cost of the original
computation, so for significant savings in computational time, expensive
calculations should be preferred.
Good candidates for this are expressions which involve the computation of
mathematical logs and exponentials.
The requirement for dependence on only one variable is due to the nature of the
table.
It must be indexed by the steps in the dependent variable or variables.
If there are $N$ steps in a dependent variable to be tabulated, adding
dependence on a second variable requires pre-computing and storing
$N\,\times\,N$ values.
The value of $N$ varies, but is typically at least 1000.

With these limitations in mind, there are still a number of expressions that can
be tabulated in the typical electrophysiological cell model.
If the Rush-Larsen method has been used to integrate gating variables then the
expressions for both steady state and time constant of the gate can typically be
tabulated.
Currents with a complex, but wholly voltage dependent form, such as \ii{K1}\ in
the Courtemanche et al. human atrial cell model can have the current calculation
tabulated.
Other calculations in a cell model must be evaluated on a case-by-case basis.

The use of lookup tables can significantly speed up code execution.
It can also influence the results, due to the discretisation involved in the
computations.
However even a small number of steps, sufficient to discretise with a
resolution of \mv{0.1}, typically indroduces a $leq$~0.1\% error.
Meanwhile, a 5x or greater speedup is not impossible.

\subsubsection{Binary Searches}

Several of the experimental protocols provided by the library are intended to
determine the value of a parameter which causes a particular condition to be
fulfilled, such as a successful excitation of the cellular model after
progressively shortening stimulus intervals.  This value we will call the
critical value. In real experiments, ones involving actual cardiac tissue, the
typical experimental protocol would involve stimulating the tissue at
sequentially shorter intervals, until no stimulation was provoked.  This might
involve stimulating the cell thousands of times, which would be expensive
computationally to model exactly.  Instead, a binary search for the critical
value can be performed, using the pseudo-code shown in Algorithm~\ref{toolkit:binary}

\begin{algorithm}
\caption{
Binary search for the critical value of the function $f(x)$.
The critical value is defined as the smallest $x$ which still makes $f(x)$
produce a value, $v$, greater than the threshold, $t$.
The initial guesses for $x$ are $high$ and $low$.
The guessing continues until sufficiently close for the accuracy condition to be
fulfilled.
}
\label{toolkit:binary}
\begin{algorithmic}
\STATE $x_{high} = high$
\STATE $x_{low} = low$
\REPEAT
\STATE $x_{current} = \left(x_{high} - x_{low}\right) / 2$
\STATE $v = f\!\left(x_{current}\right)$
\COMMENT{Compute $v$ using $x_{current}$}
\IF {$v \geq t$}
\STATE $x_{high} = x_{current}$
\ELSE
\STATE $x_{low} = x_{current}$
\ENDIF
\UNTIL{$\left(x_{high} - x_{low}\right) \leq accuracy$}
\end{algorithmic}
\end{algorithm}

To explain in words, first two guesses are made; the high guess, which is the
maximum value that the critical value can take, and the low guess, the minimum
it is presumed to take.  The simulation is then run with the parameter set at
the average of the low and high guesses--the current guess.  If the test is
successful, the critical value evidently lies somewhere between the low guess
and the average, and so the high guess is set to the current guess.  Conversely,
if the test is unsuccessful, the critical value is obviously above the current
guess, and so the low guess is set to the current guess.  The simulation is then
repeated with the average of the new high and low guess.  Using this algorithm,
the search space is halved with each iteration, swiftly finding the critical
value.
For example, to find a parameter to an accuracy of \ms{1}\ in a range of
\ms{250}\ statistically requires 125 sequential iterations.
To find that same parameter using binary iterations requires just 8.

An important limitation of the binary search method is that there must only be
one critical value.
If there are two such values within the range, the result of the algorithm is
unpredictable.
In practice, this constraint is quite easy to work within.

\subsubsection{Adaptive Step}

Adaptive step mechanisms are employed in the library when there is a need to
provide output over a wide range of times, when the slope of the graph is not
constant over the range to be graphed.  This is very common in the modelling of
cardiac cells, which often show an exponential dependence of various parameters
on the  stimulus interval, and are graphed over a range of hundreds or thousands
of milliseconds.  A step that sufficient to track the curve at the upper limits
of the range will completely fail at the steeper slow of the lower limits,
whilst a step that will track the curve for the lower limits will result in
unnecessary work being done at the upper end of the range.  To alleviate this
problem, an adaptive stepping mechanism is used, as shown in the pseudo-code
Algorithm~\ref{toolkit:adaptive}.

\begin{algorithm}
\caption{
Adaptive stepping algorithm for calculating a value, $v$, for decreasing values
of time, $t$ with the function $f\!\left(t\right)$.
$t$ starts at $t_{max}$ and is computed until $t_{min}$.
The initial step size used to reduce $t$ is $step$.
}
\label{toolkit:adaptive}
\begin{algorithmic}
\STATE $step = step_{start}$
\STATE $factor = \frac{step}{2}$
\STATE $v_{last} = f\!\left(t_{max}\right)$
\STATE $t_{prev} = t$
\STATE $t = t_{max} - step$
\WHILE{$t \geq t_{min}$}
    \STATE $v = f\!\left(t\right)$

    \IF {$|v - v_{last}| \leq threshold$}
        \PRINT $t, v$
        \STATE $v_{last} = v$
        \STATE $t_{prev} = t$
        \STATE $t = t - step$
    \ELSE
        \STATE $step = step - factor$
        \STATE $factor = \frac{factor}{2}$
        \STATE $t = t_{prev} - step$
    \ENDIF
\ENDWHILE
\end{algorithmic}
\end{algorithm}

First, the measurement is performed at the largest desired point.  The interval
is then reduced by the step, and the measurement is performed again.  The
difference in the measurements is calculated and compared to the desired maximum
delta.  If the difference is acceptable, the interval is once more reduced by
the step, and the measurement taken once more.  If the difference is too great,
then instead the step size is halved and the measurement repeated.  If the
difference is now acceptable, then the interval is reduced by the new step and
the experiment proceeds.  If it is not, then the step size is once more halved.
The step size used is therefore always appropriate to the slope of the curve and
a smooth graph results.  Additional logic, not shown in the pseudo-code, is used
to ensure the step size does not become too small, and to terminate the graph at
the lower end of the range.

Since curves can increase or decrease the absolute difference between the two
values is compared.

\subsubsection{Parallel Input/Output}

Input and output for simulations is obviously essential if they are to be of any
use.  This input and output can involve the reading or writing of many megabytes
or gigabytes of information over the course of a simulation, often in a variety
of different formats.  This most significant for sheet simulations and it is
those that we consider here.

Data input is typically not that significant a cost for
simulations, as whilst various simulation maps and saved states be quite large,
the cost is typically only paid once.  Data output is often required at numerous
points throughout the course of the simulation since almost all simulations are
performed to examine the time evolution of the cardiac system.  Many simulations
will output the value of state parameters, most commonly the voltage, at all
points in the tissue every few milliseconds of simulated time.  Other state
variables might also be desired and the toolkit also offers `live visualization'
of two dimensional sheets via output of images in the GIF format.  In addition,
the whole state is output regularly, to allow simulations to be resumed at a
later point in time.  This output usually stops all simulation whilst it is
being written to disk, time during which the processors are typically idle.  By
allowing multiple threads to output in parallel, the cost of outputting multiple
files can be reduced to the cost of outputting one.


\section{Experimental Protocols}
\label{sec:toolkit:protocols}

The toolkit developed provides a number of experimental protocols to use with
the cellular models to quantify the electrophysiological behaviour of the
modelled cells.  The provided protocols include the action potential duration at
90\% repolarisation (\apd) and the action potential (AP) profile; the \apdr\ and
\apdr[50]\ restitution; the effective refractory period (ERP) restitution
(ERP\emph{r}); the conduction velocity (CV) restitution (CV\emph{r}); the
temporal vulnerability window to unidirectional conduction block (VW); the
threshold of excitation and a flexible system for specifying two dimensional
sheet experiments, including the initiation of re-entry via wavebreak protocols
and computation of the spatial vulnerability window.

\subsection{Action Potential}

The action potential profile is one of the fundamentals of cellular modelling,
with a number of associated properties.  These include the action potential
duration at 90\% repolarisation, \apd; the action potential
duration at 50\% repolarisation, \apd[50]; the maximal overshoot, OS; the
upstroke velocity, $\frac{dV}{dt}_{\text{max}}$\ and the resting membrane
potential.

To compute these quantities, the cell is paced $N$\ times at given S1 interval.
After another S1 interval, a final AP is elicited from the cell and the
properties are measured.  In addition, it is common to want current and
cellular model state traces over the course of an AP and these can be provided
by the library alongside the membrane potential trace.

\subsection{Action Potential Duration Restitution}

The library calculates the APD\emph{r} via a standard S1--S2 protocol used in both
numerical simulations and also in physiological experiments.  The APD\emph{r} is used
as a measure of how the cell responds to stimulations at different rates.

The cellular model is paced $N-1$\ times with a stimulus close to the threshold
value at a given stimulus interval, S1.  At this point, the state is saved for the
paced cells.  The $N$th S1 stimulus is then given, followed by the S2 after a
varying DI, which is reduced via an adaptive step to record the relationship
between DI and the APD of the following AP.  The toolkit also determines useful
parameters such as the maximal slope of the restitution curve, which can be
related to the stability of spiral waves within the tissue.  Both the \apdr\ and
the \apdr[50]\ restitution can be calculated.

\subsection{Effective Refractory Period Restitution}

The ERP\emph{r} is calculated by the library using standard experimental protocols.
The ERP is defined as the shortest possible stimulus interval, S2, which still
allows a successful AP to be elicited after pacing $N$ times at a pacing
interval S1.  A successful AP is defined as an AP which has an amplitude
of at least 80\% of the magnitude of the preceeding AP.  
The rate dependence of the ERP is evaluated at a decreasing S1 interval.

To find the ERP for a given S1 interval the cellular model is paced $N$ times
at that interval.
The state is saved just before the $N-1^{\text{th}}$ AP is initiated.
The ERP is found via binary search.
The low guess for S2 is typically chosen as zero, whilst the high guess is the
S1 interval being tested.
The S2 interval for each attempt is the average of the high and low guesses.
After the state has been saved, the $N^{\text{th}}$ AP is initiated and its
amplitude recorded.
Then \ms{S2} after, the test AP is evoked.
After the test AP has been allowed to run its course, the amplitude is tested
and used to guide the next binary iteration.
Details of the elicited APs, such the S1 and S2 amplitudes and durations are
stored.
The binary iteration proceeds until the desired accuracy has been attained.

The reduction in S1 interval is stepped via an adaptive mechanism which is
used to keep the reduction in ERP between successive S1 intervals to below
\ms{1}.  The S1 interval is reduced until it is sufficiently short that the
S2 interval would fall within the $N^{\text{th}}$ AP.

\subsection{Vulnerable Window}

The VW measurement is based around a 1D ring model of cardiac tissue.
It is used to quantify the vulnerability of cardiac tissue to the genesis of
arrhythmia via re-entrant activity.
The VW is defined as the time period in the refractory tail of a propagating
excitation wave that results in unidirectional conduction block.
In the case of a ring model of the tissue this causes retrograde propagation,
which forms a re-entrant excitation which cycles endlessly.
If the stimulus is given too early, then the tissue will still be refractory in
both directions and no propagation of excitation will ensue.
If it is given too late, then propagation will occur in both directions, which
in the ring case, results in the two excitation wavefronts annihilating each
other.
Normal pacing could then resume.
These three cases are illustrated in Figure~\ref{fig:toolkit:vw}.




The VW is found in a 1D strand model, set up as described previously.
The use of a open-ended strand, not a connected ring, makes pacing the strand
easier, but does not effect the results.
The strand is 300 units long and with a space step of \mm{0.1}.
The strand is first given $N$ S1 conditioning stimuli which are administered to a
3 node (\mm{0.3}) section at the start of the strand.
Typically an S1 interval of \ms{1000}\ is used with 10 S1 pulses.
The test S2 stimulus is administered to a 4 node (\mm{0.4}) section, normally
centred in the middle of the strand, \ms{S2}\ after the $N^{\text{th}}$
conditioning excitation wave has passed the S2 stimulus site.
To reduce the computation time, the state is saved at this point.
The low guess for the binary iteration is chosen as \ms{0}\ and the high guess
as the S1 interval.
To judge the success of the S2 stimulus, the ends of the strands are watched for
successful excitation.
If there are no excitation waves crossing the ends after the $N^{\text{th}}$
excitation wave has passed, then it is in the region of total conduction block.
If one excitation wave crosses the end, it is in region of unidirectional block.
When two excitation waves cross the end, it is in the region of bidirectional
conduction.
The timing of the S2 stimulus is controlled via binary search, first to find the
boundary of total conduction block and unidirectional block and then to find the
boundary between unidirectional block and bidirectional conduction.
A minor optimisation above the usual binary search algorithm is possible in this
case, as a search for one boundary can be used to refine the range for the
second boundary too.

\subsection{Threshold of Excitation}

The threshold of excitation is a theoretical measure, proposed by Zhang et
al.~\cite{Zhang2003}\ and used in modelling studies~\cite{Kharche2008}.  It is
defined as the minimum stimulus current which, when delivered to a cell, will
cause the cell to depolarize to a membrane potential of at least \mv{-20}.  The
threshold of excitation is calculated for a range of stimulus intervals,
successively reducing the interval until it is impossible to elicit a
depolarization of sufficient magnitude.  At each stimulus interval it is
recorded if the test pulse elicits bidirectional, unidirectional or no
propagation.

The threshold of excitation is found in a 1D strand model.  The strand is 300
units long, with a space step of \mm{0.1}.  The strand is first given $N$ S1
stimuli at a rate which allows the strand to recover between each excitation
wave.  Each S1 stimulus is delivered to the first 4 nodes (\mm{0.4}) and is
chosen to be above the threshold of excitation.  The threshold of excitation is
calculated at the 100th node and so as this node depolarises, the state for the
whole strand is cached.

The threshold of excitation is then found via binary search, with a lower bound
of \unit{0}{nS} and an upper bound chosen to be 5x the normal threshold.  The
test stimulus is delivered $\Delta t$\ seconds after the 100th node depolarises
to a group of 4 nodes centred on the 100th node.  After the test stimulus is
delivered the 4 nodes are tested for the excitation condition, attaining a
membrane potential of \mv{-20}.  If it is successful, the current stimulus
strength will assigned to the high guess.  If not, to the low guess.  In
addition, the simulation is continued to evaluate whether bidirectional,
unidirectional or no conduction of the excitation wave is evoked by the
stimulus.  The strand is then reset to the cached state and the new stimulus
strength is tested until a sufficient accuracy has been attained.  Once the
threshold of excitation has been determined for a given $\Delta t$\ the state of
the strand is once more reset to the cached state and a shorter $\Delta t$\
tested.

\subsection{Conduction Velocity Restitution}

The CV is the rate of propagation of the excitation wave.
It is determined by the difference in excitation times at two points divided by
the distance between them.
It is measured both in both experimental and numerical studies and is therefore
useful in validating experimental results.
A related measurement is the minimum conduction interval.
The minimum conduction interval is the shortest interval between an S1 and S2
stimulus which still propagates successfully.
It is similar to the ERP, but can also be influenced by inter-cellular coupling
and heterogeneity in the strand.
The CV\emph{r} is found via stimulating the strand at successively shorter
intervals and noting the changes in the measured CV.
The minimum stimulus interval is found via noting when the curve ends.


The CV\emph{r} is found in a 1D strand model, set up as described previously.
The strand is 300 units long with a space step of \mm{0.1}.
Stimuli above the stimulus threshold are delivered to a 4 unit (\mm{0.4}) length
at one end of the strand.
The strand is first given $N$ S1 stimuli.
Typically an S1 interval of \ms{1000}\ is used with 10 S1 pulses.
The S2 stimulus is then delivered \ms{S2} seconds later.
The CV is estimated from the difference in excitation times, defined as the
instant at which the node is excited above \mv{-60}, at 2 nodes which are
located 100 nodes apart.
This minimizes any possible influence from boundary conditions.
The S2 time is then stepped via an adaptive step, until a second excitation wave
does not propagate the length of the strand.


\subsection{Spiral Wave Dynamics}

The dynamic behaviours of spiral waves are characterised by the stability,
mobility and lifespan (LS).
Spiral Wave LS is examined experimentally and numerically.
The LS of the spiral wave and the meander pattern of the tip are both used to
gain insight into the behaviour of the tissue under conditions of cardiac
arrhythmia.

Spiral waves are initiated in a square sheet of tissue $375\,\text{x}\,375$
nodes in dimension with a space step of \mm{0.1}, as described previously.  The
tissue is first stimulated along one edge via a stimulus current applied to a
row of nodes extending the length of the tissue and 3 nodes (\mm{0.3}) in width.
The planar wave is then allowed to propagate over the tissue.  Some time after
the first wave is initiated, a second stimulus is applied.  The second stimulus
is applied to half the tissue, bisecting the propagation front of the first wave.
The second stimulus is a voltage clamp, with all the included tissue clamped to
a `high' potential, typically \mv{+0} for a millisecond.  The
generated spiral is then allowed to evolve until it self-terminates, the spiral
wave tip exits the tissue or until a sufficient amount of time has passed such
that the spiral can be classified as `persistent'.  The time allowed for a wave
to be classified as persistent is typically 5 or \unit{10}{s}.

The spiral wave tip traces are calculated via a standard contour based
algorithm, comparing the \mv{-60} contour line on snapshots of the electrical
activity \ms{2.5} apart.

\subsection{Spatial Vulnerability of Cardiac Tissue}

The spatial vulnerability (SV) of cardiac tissue is defined as the smallest
length of tissue which, when given a stimulus at the threshold level in the wake
of a propagating wave, gives rise to at least one `figure of eight' re-entry.
A figure of eight re-entry occurs when the excitation waves from the ends of the
test length propagate back through the centre of the length.
This results in a pair of contra-rotating spiral waves, one at each end of the
test length.
The SV is useful for quantifying a mutation or condition's potential for
arrhythmogenesis by giving an indication of the size of ectopic focus required
to excite the tissue.
A small SV indicates that the tissue could be very likely to have arrhythmic
episodes.

The sheet model used for the determination of the SVW can vary in size, as the
SVW can vary substantially, depending on the electrophysiology being simulated
by the cellular models at the nodes, but the smallest used is typically
$375\,\text{x}\,375$ nodes, with a spatial resolution of \mm{0.1}.  The sheet is first given one
conditioning excitation, initiated by injecting a strip of nodes 3 nodes
(\mm{0.3}) in width with current along one edge of the sheet.  The wave is then
allowed to propagate through the tissue.  When the VW of the tissue is
positioned at the centre of the tissue, the test stimulus is delivered.  The
test stimulus is an area of tissue 20 nodes (\mm{2}) wide and of variable length.
After the test stimulus is delivered, the sheet is observed until figure of
eight re-entry is observed, or it is obvious that it will not occur.  The
protocol is then repeated with a test stimulus area of greater length.


\section{Results From Simulation Studies}

The cardiac simulation toolkit has been used in several simulation studies.
Here I present two such studies, representing different aspects of the toolkit.
The first is based on an experimental study which determined the existence of a
novel ion channel in the human atrium which caries an anion current through the
cellular membrane.  The second is principally a 2D study, concerning the effects
of Atrial Fibrillation induced Electrical Remodelling (AFER) and
electrophysiological heterogeneity in a 2D idealization of the right atrium and
the sino-atrial node.

\subsection{Anion Currents In The Human Atrium}

\subsubsection{Introduction}

In a recent experimental study, Li et al.~\cite{li2007} determined the existence
of a novel outwardly rectifying anion current in human atrial myocytes isolated
from right atrial appendages taken from patients undergoing coronary bypass
surgery.  They determined that it was separate other known ionic currents and
performed preliminary modelling based on the CRN Human Atrial Myocyte
model~\cite{CRN98}.  They determined the current could be modelled by
equation~\ref{anion:eqn} with the constants given in table~\ref{anion:table}.

The total current carried by the channel, \ii{ANION}, was given by

\begin{equation}
\label{anion:eqn}
I_{ANION} = g_{ANION} \frac{V-E_{ANION}}{1-\left(c\times e^{\left(V-E_{ANION}\right)}\right)}
\end{equation}

where $g_{ANION}$ is the conductivity of the anion channel, $E_{ANION}$ is
the reversal potential of the channel and $c$ and $d$ are constants to
describe the behaviour.  All other symbols have their usual meanings.

\begin{table}
    \caption[Parameter sets for the anion sensitive current]{
        Parameter sets for the anion sensitive current \ii{ANION}\ when carrying
        \nothree\ and $\text{Cl}^{\text{-}}$\ ions.
    }
    \begin{tabular}{ l  c c}
    \label{anion:table}
    & \nothree & $\text{Cl}^{\text{-}}$ \\
    \hline
    $g_{ANION}$ & 0.37   & 0.19 \\
    $E_{ANION}$ & -45.64 & -45.64 \\
    $c$         & 0.87   & 0.94 \\
    $d$         & $8.4\,\text{x}\,10^{\text{-4}}$ &  $2.5\,\text{x}\,10^{\text{-4}}$
    \end{tabular}
\end{table}


The simulation study used the parameter set for the anion current carrying
\nothree ions.  The effects of the addition of this current to atrial
myocyte cells was quantified .  In the following
paragraphs, `control' is used to denote the unmodified CRN model and `anion' to
denote the CRN model with the additional current described by (\ref{anion:eqn})
with the \nothree\ parameter set from Table \ref{anion:table}.

\subsubsection{Methods}

The effect on the behaviour of the cells caused by the introduction of the anion
current was quantified using the simulation library described previously in this
chapter for control (no \nothree-sensitive \ii{ANION} current) and anion
(\nothree-sensitive \ii{ANION} current present) cases.  As this simulation study
was based on the CRN cell, the standard stimulus was \ms{2} in duration and
\unit{2}{nS} in magnitude.  Unless an alternative protocol is mentioned, all
simulations directly followed those set out at the start of this chapter.  None
of the simulations presented in this section involve the use of lookup tables of
voltage dependent properties.

Simulating a single cell, the following measures were quantified: the AP
profile, the restitution of APD at 50\% repolarization, \apdr[50], the
restitution of APD at 90\% of repolarization, \apdr\ and the Effective
Refractory Period restitution, ERP\emph{r}.  The maximal fast sodium activation
was quantified at the same time as the \apdr\ was computed and is the product of
the three gates in \ii{Na}, as $m^{3}hj$. In all the single cell cases, the
cell was paced 10 times before the measurement was taken, to allow simulation
parameters to settle and to adapt to any changes in pacing rate.  Storage of the
cellular state was used in all appropriate points in the simulation, to minimise
computational time.

Using a 1D strand model the temporal Vulnerability Window to unidirectional
conduction block, VW, the Conduction Velocity restitution, CV\emph{r}\ and the
threshold of excitation were computed.  The strand model used was 300 nodes long
and had a space step of \mm{0.1}.  The diffusion coefficient, $D$, was set to
$0.03125\,\text{mm}^{\text{2}}\,\text{ms}^{\text{-1}}$~\cite{Biktasheva2005}.
In all 1D simulations the strand was paced 10 times before measurement was
taken.  In all simulations this state was then cached and restored as
appropriate, as described in the algorithms section of this chapter.

Using a 2D tissue model the lifetime of re-entrant spiral waves was estimated,
following the wave-break protocol outlined earlier.  The sheet had dimensions of
$375\times375$ nodes and a space step of \mm{0.1}.  The clamp potential used
to break the wave was \mv{+50} and it was applied for \ms{1}.

\subsubsection{Results}

The AP generated by the control and anion simulations are shown in
figure~\ref{anion:ap}.  The \apd\ is relatively unchanged, but is slightly
reduced from \ms{299.6} to \ms{298.0}, whereas the \apd[50]\ is significantly
reduced, from \ms{180.1} to \ms{160.1}.  The AP profile shows a depressed plateau region (phase
2), reduced from \mv{-9.56}\ to \mv{-14.1}\ and a slightly elevated resting potential,
\mv{-79.0}\ in the anion case cf. \mv{-80.9}\ in control.  This duplicates the Li et
al.~\cite{li2007} study, and is included here only for completeness.

\begin{figure}
\includegraphics{figures/toolkit/anion/01_AP}
\caption[Anion Sensitive AP Profile]{\label{anion:ap} AP profile for the CRN
model in control (black) and anion (red) cases.  The inclusion of the
\nothree\ carrying current results in a small change of AP morphology, with
a depressed plateau potential and an elevated resting potential.}
\end{figure}

The APD\emph{r}\ curves produced by the control and anion cases are shown in
figure~\ref{anion:apdr50} and \ref{anion:apdr90}, for the restitution of APD at
50\% (\apdr[50]) and 90\% (\apdr) of repolarization, respectively.  The
\apdr[50]\ shows the most significant differences, with the anion
curve depressed by \ms{20} even at the largest DI, increasing to a maximum
difference of over \ms{40} at at DI of \ms{380}.  The two curves then rejoin each
other before they cross over at a DI of \ms{200}.  The \apdr\ curves, by
contrast, are very similar for control and anion cases at large (over \ms{600}) DI.
Between \msrange{100}{400}\ DI, the anion case is depressed compared to the control
case, with a difference of up to \ms{25}\ observed in the measured APDs.  At
\ms{100}, the curves rejoin one another and show a rapidly increasing slope as the
DI approaches \ms{0}.

\begin{figure}
\includegraphics{figures/toolkit/anion/03_S1S2_50}
\caption[Anion Sensitive APD Restitution at 50 repolarization]{
\label{anion:apdr50} APDr curves for the CRN model in control (black) and anion
(red) cases at 50\% repolarisation.  The two variants are clearly different for
all the DI tried in the simulation, with the anion case considerably below the
control case for much of the DI tried.  The anion case also shows a reduced
slope of the restitution curve compared to the control case.  Note the
cross-over of the curves at a DI of \ms{200}.}
\end{figure}

\begin{figure}
\includegraphics{figures/toolkit/anion/02_S1S2_90}
\caption[Anion Sensitive APD Restitution at 90 repolarization]{
\label{anion:apdr90} APDr curves for the CRN model in control (black) and anion
(red) cases at 90\% repolarisation.  The two variants behave the same at large
DI, but as the DI before the S2 stimulus is decreased, the anion case shows a
greater reduction in the \apd.  At short DI (below \ms{100}) the two curves
rejoin each other.}
\end{figure}

The ERP\emph{r} curves produced by the control and anion cases are shown in
figure~\ref{anion:erpr}.  In both cases the ERP\emph{r}\ curves are relatively
flat, decreasing by approximately \ms{60}\ over the \ms{700}\ range of S1
intervals considered.  The addition of the \ii{ANION}\ current changes the
behaviour of the ERP\emph{r} curve in a manner which is not simply a shift left
or right.  The control case shows a response which has a clear plateau region
which continues until an S1 interval of \ms{500}\ is reached and then a
relatively steeper decline until eliciting an AP of the appropriate magnitude
becomes impossible at an S1 interval of \ms{330}.  The anion case, by contrast,
shows a decreasing ERP over the whole range of S1 intervals considered although
it too shows its steepest slope just before eliciting a sufficiently large AP
becomes impossible, also at approximately \ms{330}.  At long S1 intervals (above
\ms{700}) the ERP\emph{r}\ is longer for the anion case before the curves cross
at \ms{600}\  and then again at \ms{450}\ with the ERP in anion at the point
where further stimulation becomes impossible almost \ms{20}\ higher than in the
control case.

\begin{figure}
\includegraphics{figures/toolkit/anion/04_ERPR}
\caption[Anion Sensitive Effective Refractory Period Restitution]{
\label{anion:erpr} ERP\emph{r} curves for the CRN model in control (black) and anion
(red) cases.  The
addition of the \ii{ANION} current changes the behaviour of the cell from a long
and flat plateau region followed by a relatively sharp decrease into a more
constant decline.}
\end{figure}

The maximal activation of the fast sodium current, \ii{Na}, is shown in
figure~\ref{anion:ina}.  The presence of \ii{ANION}\ consistently reduces the
maximal activation of \ii{Na}.  At long S1 intervals, greater than \ms{500}, the
reduction is 3\%, increasing to almost twice that in the range of
\msrange{400}{450}.  Both curves rapidly decrease to almost no \ii{Na}\
activation at an S2 interval of \ms{300}\ but the anion case starts this descent
first.

\begin{figure}
\includegraphics{figures/toolkit/anion/08_INa}
\caption[Anion Sensitive maximum activation of the fast sodium current.]{
\label{anion:ina} Maximal activation of the fast sodium current, \ii{Na}, as S1
interval is decreased.  The fast sodium current consistently activates to a
greater degree in control cases than in anion cases.  Note also the rate
dependent nature of the effect, with the greatest difference observed at S1
intervals of \msrange{400}{500}.}
\end{figure}

The temporal VW increased with the addition of the anion current from
\ms{3.20} in
control to \ms{3.81}\ in anion case, a 20\% increase in the size of the region of
unidirectional conduction block.  The CV\emph{r}\ curves, shown in
figure~\ref{anion:cvr}, suggest that tissue with the anion sensitive current
shows faster CV at normal physiological stimulus intervals (corresponding to
\msrange{500}{1000}), with an average conduction velocity of
\cms{27.2}\ in anion, compared with \cms{26.9}\ in control.  As the conduction
interval is reduced below \ms{500}, the conduction velocity starts to decrease
rapidly until conduction stops at \ms{325} for anion and \ms{319} for control.
There is a brief recovery of conduction velocity visible in both cases, just
before conduction block.

\begin{figure}
\includegraphics{figures/toolkit/anion/06_CV}
\caption[Anion Sensitive Conduction Velocity Restitution]{
\label{anion:cvr} CV\emph{r}\ curves for the CRN model in control (black) and anion
(red) cases. The CV\emph{r}\ curves are relatively flat for both cases over the range of
\msrange{500}{1000}, before they both increase rapidly in steepness until the minimum
stimulus interval is reached at approximately \ms{320} for both control and anion.
The CV is higher at longer stimulus intervals for the anion case, before it
crosses the control curve at an approximate stimulus interval of \ms{460}.}
\end{figure}

The threshold of excitation, shown in figure~\ref{anion:toe}, shows that
\ii{ANION} reduces the minimum stimulus current by approximately \unit{0.2}{nS},
a reduction of 10\%, at almost all $\Delta t$\ intervals.  It is also
interesting to note that both control and anion tissue types show `supra-normal'
excitability, with the minimum stimulus current decreasing as $\Delta t$\
decreases.  This occurs until a critical point is reached when the cell suddenly
becomes significantly harder to excite, with the minimum stimulus current
increasing almost 100\%.  This occurs \ms{9}\ later in control tissue at
\ms{321}\, compared with \ms{330}\ in control.


\begin{figure}
\includegraphics{figures/toolkit/anion/09_ToE}
\caption[Anion Sensitive Threshold Of Excitation]{
\label{anion:toe} The threshold of excitation curves for the CRN model in
control (black) and anion (red) cases.  Both curves show a decreasing threshold
of excitation until a critical point is reached and the amount of current that
must be injected to reach the threshold voltage is almost doubled.  In the
control case this comes at an $\Delta t$ of \ms{321}\ and in anion, it comes at
\ms{330}.  Until this critical point is reached, the threshold of excitation is
consistently lower for cells with \ii{ANION} compared with control.
}
\end{figure}

Spiral waves were induced in a square sheet.  Representative plots of the
membrane potential over the whole sheet, produced as the simulation was ongoing,
are show in figure~\ref{anion:spiral}.  Panels A i--iv show the membrane
potential for control, whilst Panels B i--iv show the membrane potential for
anion, as the simulation evolved.  The path followed by the spiral wave tip as
it meandered is shown in A,v and B,v for control and anion, respectively.  In
both cases the spiral wave starts in the centre of the tissue and then follows a
looping track around the tissue before finally it exits the tissue when it
cannot turn fast enough around its own refractory tail.  This process takes
\ms{1700} in control and \ms{2300} in anion.

\subsubsection{Discussions and Conclusions}

The effects of the inclusion of an anion sensitive current do not seem to be
that large, at least when considered on the single cell level.  This is perhaps
to be expected, as a current which did have a significant influence on the
action potential would surely have been identified sooner.  However, despite the
current's small influence on the action potential duration, it does have
significant effects on the restitution properties of the cell and on the
behaviour of cells in a tissue.

The most noticeable effect of the inclusion of \ii{ANION}\ in a cellular model
is the abbreviation of the \apd[50]\ and the accompanying reduction in the
plateau potential.  The abbreviation is due to \ii{ANION}\ acting as a
rectifying current when the membrane potential is above \mv{-45}.   Conversely,
at potentials below \mv{-45}\ \ii{ANION}\ acts to depolarize the cell, leading
to the slightly elevated resting membrane potential observed between action
potentials.  This difference in effect is what leads to the interesting
behaviours observed in cells with \ii{ANION}.

The \apdr[50]\ and \apdr\ curves show that \ii{ANION} has a rate dependent
effect.  Both curves are flattened in the cells which include \ii{ANION}\, but
this flattening is not uniform over the range of S2 intervals considered.
\ii{ANION}\ has a simple exponential dependence on the membrane potential and no
gating variables however, so it is not \ii{ANION}\ which causes this rate
dependence directly.  Instead, we must look to the currents active within the
plateau region of the action potential.  \ii{CaL}\ is the principle current
responsible for the plateau region of the action potential and unlike \ii{ANION}
it has both an activation gate, $d$, and an inactivation gate, $f$.  The $d$\
gate is not as interesting as the $f$\ gate, as its time-course is not affected
by the presence of \ii{ANION}, although its activation during the plateau region
is reduced.  Conversely, the $f$\ gate in \ii{ANION} cells never inactivates as
completely as it does in the control simulations which lack the current.

Considering next the 1D strand results, both the CV\emph{r} and threshold of
excitation data continue the story of a rate dependent influence.  At a long
stimulus interval, the increased excitability of the anion cells leads to a
higher conduction velocity.  The increased excitability at long stimulus
interval is due to the inward nature of the current in the very first stages of
the action potential.  This increased excitability allows the cells with
\ii{ANION}\ to conduct excitation faster until, at stimulus intervals of below
\ms{500}\ control cells start to conduct faster.  At this stimulus interval, the
threshold of excitation is still lower for the anion case, so another factor is
responsible for the reduction in conduction velocity.  The excitability of the
cell is an important influence on the conduction velocity, but it is not the
only factor.  Another major factor is the upstroke velocity which is principally
governed by the fast sodium current, \ii{Na}.  This is partially inactivated by
the elevated resting potential in the anion case, which also reduces the rate of
recovery of the inactivation variables.  When the test stimulus is delivered
after a reduced conduction interval in the anion case \ii{Na}\ does not open as
fully, slowing the upstroke and thus leading to a reduced conduction velocity at
short stimulus intervals, compared with the control case.  The increase in the
vulnerability window appears to be quite significant, an extra 20\% of the size
of the vulnerability window in tissue without \ii{ANION}.  Both the reduced
vulnerability window and the reduced conduction velocity at short pacing
intervals suggests that the addition of \ii{ANION} may be pro-arrhythmogenic.
The increased vulnerability window has an obvious influence on the genesis of
re-entrant excitation---A larger vulnerability window increases the chance of a
premature excitation interrupting the normal function of the heart.  The
influence of conduction velocity on re-entrant excitation is subtler, but a
reduced conduction velocity reduces the wavelength of the excitation wave.  This
allows a greater number of excitation waves to persist in the same area of
tissue, reducing the likelihood of a re-entrant excitation self-terminating.

Finally, considering the results from the sheet model of the tissue which was
used to investigate the lifespan of induced spiral wave re-entry, we can see our
hypothesis from the 1D strand results is born out.  A revolving spiral wave
typically rotates at a higher frequency than the normal excitation rate of
cardiac tissue.  The excitation waves in the 2D sheet are therefore operating in
the short stimulus interval regime.  Cells which contain \ii{ANION}\ have a
slower conduction velocity in this regime and so the spiral wave can be
sustained for longer due to it's shorter wavelength.  However, the reduction in
wavelength is not so severe as to create a stable mother rotor, as been observed
in studies of pathological conditions.

\subsection{Atrial Fibrillation Induced Remodelling And Heterogeneity}

\subsubsection{Introduction}

The human atria consists of several tissue types each with distinct
electrophysiological properties.  It has previously been shown that
inhomogeneity in tissues can lead to re-entrant activity
\cite{Bernus2005, Coronel1992, Kumagai1997}.  There is also experimental
data available on the ion channel remodelling due to atrial
fibrillation induced remodelling (AFER) during chronic atrial
fibrillation (AF) on human atrial cells~\cite{Bosch1999,Workman2001}.

In this study, we quantified the changes in electrophysiological
behavior in cell and 1D homogeneous models under AFER compared to control
conditions.  Further, re-entrant waves in 2D electrically homogeneous
and electrically heterogeneous sheets were studied.

\subsubsection{Methods}

The human atrial action potential (AP) model by Courtemanche et
al.\cite{CRN98} was used in this study.  Modifications were
incorporated to reproduce the differing APs of the different atrial cell
types~\cite{Seemann2006}.  This produced distinct APs for the
crista terminalis (CT), pectinate muscles (PM), atrio-ventricular ring
bundle and the Bachmann bundle.  Atrial myocyte (AM) cells were modelled
by the original CRN model.

The data for AFER were taken from experiments by Bosch et
al.~\cite{Bosch1999} and Workman et al.~\cite{Workman2001}, representing
the changes in ion channels in patients after one month (AF1)
and up to six months (AF2) of chronic AF, respectively.  The
modifications to the cellular electrophysiology were described in
Kharche et al.~\cite{Kharche2007}.

The effects of AFER were quantified through a variety of measures.  The
\apdr\ and the \apdr[50]\ were calculated
as described in .  There were 9 S1 stimuli at a frequency of
\unit{1}{Hz}, followed by a varying DI.  The ERP\emph{r}
was calculated as described previously, with 7 S1 delivered at the given pacing
rate and then a final S2 stimulus was used to determine the ERP after Workman et
al.~\cite{Workman2001}.  The VW and the CV\emph{r} were determined for control,
AF1 and AF2 conditions, for each of the three atrial cell types classified by
Seemann et al.. There were therefore nine 1D strand models tested.  The strand
models were each 200 nodes long, with a spatial resolution of \mm{0.1}.  The
diffusion constant used for all simulations was set to
$0.03125\,\text{mm}^{\text{2}}\,\text{ms}^{\text{-1}}$~\cite{Biktasheva2005},
giving a solitary wave conduction velocity of
$0.267\,\text{mm}\,\text{ms}^{\text{-1}}$\ in control atrial tissue.  In all 1D
strand simulations there was 1 S1 pulse and one S2 pulse.  This pulse was
applied over 4 nodes (\unit{0.4}{mm}), had duration \ms{2} and magnitude
\unit{10}{nS}.

Further, a 2D electrically heterogeneous sheet model was developed based on a
laboratory photograph of the right atrium.  The photograph was digitized at a
spatial resolution of \mm{0.1}. The model developed by segmenting areas of the
tissue into AM, PM and CT tissue types.  The complete model had an approximate size of
$130\times100\,\text{mm}$ and consisted of approximately 1 million active cell nodes.  The
simulations were performed with all cells under control, AF1 and AF2 conditions,
with the conditions applied uniformly to the tissue.  All 2D sheet simulations
were performed with the space step of \mm{0.1} and a time step of \ms{0.05}.

\subsubsection{Results}

Simulations were performed for all three cases: control, AF1 and AF2.
However the results for AF1 and AF2 were qualitatively similar, although
AF1 showed a much more profound effect on \apd\ reduction.

Incorporating the heterogeneity and AFER data causes significant
differences in \apd\ to manifest, as shown in Figure~\ref{fig:apdr}. The
effects of AFER are not uniform across the different cell types of the
atrium and it increases the difference in \apd\ between normal atrial
myocytes and the CT cells, increasing from \ms{20.1} in control to \ms{28.1}
in AF1 and \ms{33.4} in AF2.

The APD\emph{r} curves, shown in Figure~\ref{fig:apdr}, are flatter over much
of the range of diastolic intervals for AF1 and AF2 as compared to
control.  However, the maximal slopes
of the APDr curves for CT cells are higher for AF1, 5.3, and AF2, 2.8,
compared with 2.1 in control.  The difference in the maximal slopes in
different tissue types is increased, with CT having a larger maximal
slope in all cases.  In the control tissue, the difference in maximal
slopes was 0.1, compared with 0.4 in AF1 and 0.5 in AF2 tissue.

\begin{figure}[tb]
\centering
\includegraphics{figures/toolkit/afer/2_apdr}
\caption[AFER APDr curves]{APDr curves for control CT (A), control AM/PM (B), AF1 CT (C),
AF1 AM/PM (D), AF2 CT (E), AF2 AM/PM (F). In all cases the CT action
potentials are above the AM/PM action potentials over the whole of the
range considered.}
\label{fig:apdr}
\end{figure}

In AF tissue, the ERP\emph{r} was flattened for all tissue types compared with
the control cells, as shown in Figure~\ref{fig:erpr}.  The curves also extended
to lower BCLs for AF tissue, indicating that it was possible to excite AF tissue
successfully at a higher rate than was possible in control tissue.
Heterogeneity in ERP\emph{r} was largely unaffected by AF.

\begin{figure}[tb]
\centering
\includegraphics{figures/toolkit/afer/3_erpr}
\caption[AFER ERPr curves]{ERPr curves for control CT (A), control AM/PM (B), AF1 CT (C),
AF1 AM/PM (D), AF2 CT (E), AF2 AM/PM (F).  Both AF1 and AF2 have a
significantly reduced ERPr over the whole range considered, with AF1
having a lower ERP then AF2.  In addition, AF1 and AF2 cells are still
excitable after pacing at \ms{100} shorter BCL.}
\label{fig:erpr}
\end{figure}

Conduction velocity, shown in Figure~\ref{fig:cvr}, was slowed by AF,
reducing the solitary wave velocity from $0.27\,\text{mm}\,\text{ms}^{\text{-1}}$\ in
control to $0.25\,\text{mm}\,\text{ms}^{\text{-1}}$\ in AF1 and
$0.26\,\text{mm}\,\text{ms}^{\text{-1}}$\ in
AF2.  Maximal pacing rate increased from the control value of \unit{198}{bpm} to
\unit{421}{bpm} in AF1 strands and \unit{315}{bpm} in AF2 strands.

The VW was reduced by AF, but in all cell types the reduction was small.
The control value of \ms{16.6} was reduced to \ms{14.2} in AF1 and \ms{14.7}
in AF2 for AM and PM cell types.  The reduction in VW for CT cells was
even smaller, from \ms{15.1} in control to \ms{14.5} and \ms{14.4} in AF1 and
AF2, respectively.

\begin{figure}[tb]
\centering
\includegraphics{figures/toolkit/afer/4_cvr}
\caption[AFER CVr curves]{CVr curves for control CT (A), control AM/PM (B), AF1 CT (C),
AF1 AM/PM (D), AF2 CT (E), AF2 AM/PM (F).  In each of the conditions,
the CT cells showed a higher CV at long (1000~ms) S2 intervals, but AM
and PM cells allow faster conduction at shorter S2 intervals.  The AF
cases show reduced CV compared to the control and support a higher
pacing rate via reduced minimum interval.}
\label{fig:cvr}
\end{figure}

Simulations over the 2D geometry examined the lifetime and behavior of
spiral waves in the presence and absence of electrical heterogeneity.
As can be seen in Figure~\ref{fig:plots}, panels Ai and Bi, re-entrant
activity self-terminated in both homogeneous and heterogeneous cases
when due to spiral wave meander over a large tissue region, it
exits the tissue.  Self-termination was much more rapid in the
electrically heterogeneous case, taking \unit{1.31}{s}, compared with
\unit{3.20}{s} in
the homogeneous case.

Conversely, under AF conditions the re-entry persisted after it was
induced for the whole period of the simulation, a lifespan of over \unit{5}{s}.
Under electrically homogeneous conditions, panels Aii and Aiii show a
stable mother rotor rotating anti-clockwise in the tissue.  But in
heterogeneous conditions, as shown in panels Bii and Biii, a similar
mother rotor to the homogeneous cases is visible towards the right of
each frame.  On the left of the frames, the rotor breaks up into
multiple fibrillatory wavelets on the border of the heterogeneous
regions, forming a complex and chaotic pattern of excitation.

\begin{figure*}[t]
\centering
%\includegraphics{figures/figure4}
\caption[AFER 2D re-entry plots]{Simulation of re-entry in 2D sheets of electrically homogeneous
(A) and electrically heterogeneous (B) sheets.  Columns show
representative frames after initiation of re-entry at t = 0.  Panels i
show data under control conditions, panels ii under AF1 conditions and
panels iii under AF2 conditions.  Re-entry self-terminated under control
conditions in both homogeneous (Ai) and heterogeneous (Bi).  Under AF
conditions, re-entry becomes a sustained mother rotor in
electrically homogeneous conditions (Aii, Aiii).  However, under
electrically heterogeneous conditions AF causes re-entry to degenerate
into erratic propagations on the borders of the heterogeneity (Bii,
Biii).}
\label{fig:plots}
\end{figure*}

\subsubsection{Discussion and conclusions}

AFER induces significant changes in the cellular electrophysiology that
appear to affect rate dependent electrical activities.  It helps to
sustain re-entry, providing evidence to substantiate the hypothesis of
``AF begets AF''.

Considering first the single cell results, one of the most obvious
effects is the striking reduction in the \apd\ and repolarization
properties.  AFER abbreviated \apd\ in AM cells by 66~\% in AF1 and
53~\% in AF2.  Other work has already suggested why the increases shown
in the maximal slope of the APDr can be
pro-arrhythmic~\cite{ByungSoo2002}, as can the reduced
ERP~\cite{Xie2002}.  Our study suggested that reduction is not uniform
across all cell types, which leads to an augmented heterogeneity.

The 1D strand results for the CV tell a similar story.  AFER tissue
forms a much better substrate for arrhythmic activity, supporting both a
much higher maximal pacing rate and in addition, the reduction in
conduction wavelength, to below half the control values. This allows a
greater number of excitation waves to exist in the tissue at any one
time.

The 2D simulations in the realistic sheet show a marked difference in
re-entrant behavior between homogeneous and heterogeneous simulations.
The homogeneous sheets show self-termination of re-entry in control
tissue, whilst the reduced ERP and conduction wavelength allow the rotor
to remain stable and persist for the duration of the simulation in AFER
condtions.  The heterogeneous sheet simulations, show spiral wave
breakup, as observed in real tissue \cite{Kumagai1997}, in both control
and AF simulations, possibly due to elevated plateau potentials and
increased refractory period of the CT cells.  Self-termination is still
observed in control simulations and is more rapid than in homogeneous
tissue.

It is still unclear about the pro- or anti-arrhythmogenic effects of
electrical heterogeneity in the human atria.  Self-termination is more
rapid in the heterogeneous tissue for the control case, but despite AFER
increasing the heterogeneity between tissue types, it doesn't lead to
self termination of the re-entry.  In fact, it leads to breakup of the
spiral wave in the region of the heterogeneity, leading to a region of
erratic propagations, as has been seen in experiment~\cite{Kumagai1997}.
Further study, in both 3D geometries and physiological experiments,
would be needed to elucidate the true effects of the heterogeneity.


