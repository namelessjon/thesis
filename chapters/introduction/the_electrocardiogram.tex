\section{The Electrocardiograph}

The electrocardiograph, or ECG, was developed by Einthoven and colleagues at the
turn of the 20th century.
The Einthoven ECG used the string galvanometer, developed by Einthoven, to
record the potential differences between three sets of electrodes, or leads.
These electrodes were placed on each arm and on the left leg.
These three electrodes form the bases of many ECGs recorded to this day.
The string galvanometer was highly sensitive electrical recording device for its
time, and was developed by Einthoven.
For the development of the ECG and the string galvanometer, Einthoven was awarded
the nobel prize in 1924~\cite{Kligfield2002,TEXTBOOKS}.

Since Einthoven's day, the ECG has been continually refined.
This refinement has included both improvements in the way that individual leads
are measured, and new leads that are recorded.
In addition, there have been a number of specialised lead sets developed for
particular purposes, such as exercise recording and 24 hour measurement.
Research into new lead sets, to better understand the functioning of the heart,
continues to this day~\cite{PAPERS}.

\subsection{Lead Theory}

In Einthoven's original conception of the heart and the electrical field it
produced, he developed the `Einthoven Triangle'.
The Einthoven triangle is an equilateral triangle.
At its corners sit the three electrodes and the sides of the triangles are the
leads themselves.
At the centroid of the triangle sits the heart, which is represented by a
single, stationary, time varying dipole.
The potentials measured at the three electrodes are the potentials assuming the
system is two dimensional, homogeneous and infinite in extent.
This is not entirely incorrect.

Modern lead theory was developed in the 1950's.
In a series of three papers McFee and
Johnston~\cite{McFee1953,McFee1954a,McFee1954b}\ set out their concept of the
`lead field' which is still considered applicable today.
This theory, which was a further generalisation of work by Burger and van
Milaan, allows for a heart which consisted of distributed dipole sources
sitting in a three dimensional, finite and inhomogeneous medium.

First, it is useful to define what a lead is.
A lead is defined as a pair of terminals, connected to any number of electrodes
on the body, either directly or through any number of resistors or amplifiers.
One of the terminals is designated (arbitrarily) as the `positive' terminal.
The terminals are connected such that a positive measurement is taken, should
the `positive' terminal have a higher potential than the `negative' one.

The lead theory is based on the fundamental principles of linear volume
conductors set out by Helmholtz in the middle of the nineteenth century.
These principles only hold if the body can be treated as a linear volume
conductor.
This is a good approximation~\cite{REFS}.

The first principle, that of superposition, states that the electric field
resulting from several sources in the medium is equal to the sum of the fields
which would be produced by each source considered alone.
The second principle, that of reciprocity, concerns current flow in the
medium.
It states that the current flow between two electrodes evoked by a field source in the
medium is the same as the current flow through the source evoked by placing a
potential difference across the two electrodes equal to the potential difference
that would have been created by the field source.
That is to say that the current flow is independent of the direction of
energisation, from within or outside the medium.

The concept of the lead field comes out of these principles.
To derive the lead field, we consider a unit current flowing in a lead.
The resulting flow of current through the body will have a certain magnitude and
direction at every point; it is a vector field.
Calling the vector field evoked by the unit current $\vec{J}$.

\subsection{The Twelve Lead Electrocardiogram}

The twelve lead electrocardiograph is the initial basis of almost all cardiac
diagnosis.
It started out as the three einthoven leads.
The electrodes are located (figure~\ref{fig:intro:ecg:leads}) on the left
shoulder, L, the right shoulder, R, and the feet (typically the left leg), F.
Lead I (\ref{eqn:intro:leads:i}) uses R as the negative terminal and L as the
positive terminal.
Lead II (\ref{eqn:intro:leads:ii}) is formed between R as the negative terminal
and F as the positive terminal.
Lead III (\ref{eqn:intro:leads:iii}) is formed between L as the negative terminal
and F as the positive terminal.

Wilson~\cite{Wilson1933}, in the 1930s, introduced an indifferent
electrode,  constructed by averaging the potentials at the three limb
electrodes (\ref{eqn:intro:leads:wct}).
This is the Wilson's central terminal (WCT).
They introduced three new `unipolar' leads, all of which use the WCT as the
negative terminal.
For the positive terminal, VL uses the L electrode, VR the R electrode and VF
the F electrode.
Later Goldberger~\cite{Goldberger1942}\ noted that be removing the electrode used
as the positive terminal from the calculation of the central terminal for the
negative electrode, the amplitude of the lead would be 50 per cent larger than
that of the normal unipolar leads.
These leads were termed the augmented unipolar leads, denoted by the prefix of
an `a'.
The aVL (\ref{eqn:intro:leads:avl}) leads uses L for the positive terminal and
the average of R and F for the negative terminal.
The aVR (\ref{eqn:intro:leads:avr}) leads uses R for the positive terminal and
the average of L and F for the negative terminal.
The aVF (\ref{eqn:intro:leads:avf}) leads uses F for the positive terminal and
the average of R and L for the negative terminal.
The set of leads consisting of I, II, III, aVL, aVR and aVF are known as the
limb leads.
The superior limb leads are I, aVL, aVR and the inferior are II, III, aVF.

The precordial leads were introduced by Wilson~\cite{Wilson1944}\ to provide a
better view of the electrical activity of the heart from the chest.
They are all unipolar leads which use the WCT for the negative
terminal.
For the positive terminal they use one of the six precordial electrodes
(figure~\ref{fig:intro:ecg:leads}), the locations of which are described in
many textbooks (e.g.  \cite{Hampton2008}).
The first precordial electrode, which is the positive terminal of
$\text{V}_{\text{1}}$ (\ref{eqn:intro:leads:v1}) is located to the right of the
sternum, in the fourth intercostal--between the ribs--space.
The second precordial electrode, which is the positive terminal of
$\text{V}_{\text{2}}$ (\ref{eqn:intro:leads:v2}) is located to the left of the
sternum, in the fourth intercostal space.
The third, the positive terminal of $\text{V}_{\text{3}}$
(\ref{eqn:intro:leads:v3}) is located between the second and fourth precordial
electrodes.
The fourth ($\text{V}_{\text{4}}$, \ref{eqn:intro:leads:v4}) is located on the
left midclavicular line, in the fifth intercostal space.
The fifth ($\text{V}_{\text{5}}$, \ref{eqn:intro:leads:v5}) is located on the
left anterior axillary line, in the fifth intercostal space.
The sixth ($\text{V}_{\text{6}}$, \ref{eqn:intro:leads:v6}) is located on the
left positerior axillary line, in the fifth intercostal space.

The voltage, $V$, across each lead can be written as
\begin{subequations} \label{eqn:intro:leads}
\begin{align}
V_{I}  = &\phi_{L} - \phi_{R}\label{eqn:intro:leads:i}\\
V_{II}  = &\phi_{R} - \phi_{F}\label{eqn:intro:leads:ii} \\
V_{III}  = &\phi_{L} - \phi_{F}\label{eqn:intro:leads:iii}\\
V_{WCT}  = &\frac{\phi_{L} + \phi_{R} + \phi_{F}}{3}\label{eqn:intro:leads:wct}\\
V_{aVL} = &\phi_{L} - \left(\frac{\phi_{R} + \phi_{F}}{2}\right) \label{eqn:intro:leads:avl}\\
V_{aVR} = &\phi_{R} - \left(\frac{\phi_{L} + \phi_{F}}{2}\right)\label{eqn:intro:leads:avr} \\
V_{aVF} = & \phi_{F} - \left(\frac{\phi_{R} + \phi_{L}}{2}\right)\label{eqn:intro:leads:avf}\\
V_{1} = & \phi_1 - V_{WCT} \label{eqn:intro:leads:v1}\\
V_{2} = & \phi_2 - V_{WCT} \label{eqn:intro:leads:v2}\\
V_{3} = & \phi_3 - V_{WCT} \label{eqn:intro:leads:v3}\\
V_{4} = & \phi_4 - V_{WCT} \label{eqn:intro:leads:v4}\\
V_{5} = & \phi_5 - V_{WCT} \label{eqn:intro:leads:v5}\\
V_{6} = & \phi_6 - V_{WCT} \label{eqn:intro:leads:v6}
\end{align}
\end{subequations}
where $\phi_x$ is the potential measured at electrode $x$.

As there are only 9 electrodes in the twelve lead ECG, there are only 8
potential differences which can be uniquely determined.
These are, by convention, leads I, II and $\text{V}_{\text{1--6}}$.
The value of the other limb leads is that they provide a look at the activity of
the heart from different angles, thus what might be unclear on one lead can be
obvious on another.
This concept of lead angles creates what is known as the hexaxial reference
system (\cite{Lipman1994}, pp 94., amongst others), illustrated in
figure~\ref{fig:intro:ecg:hex}.
The angle at which each lead points is the direction of the positive terminal.
Lead I, which is nominally horizontal, is at \degr{+0}.
Under the hexaxial system, Lead II has an angle of \degr{+60}\ and lead III,
\degr{+120}.
The unipolar limb leads (both augmented and not) have angles of \degr{-30}\ for
aVL, \degr{+90}\ for aVF and \degr{-150}\ for aVR.

\subsection{The ECG Waves}

In terms of the ECG, a `wave' is a deflection from the baseline observed in the
lead.
There are five standard waves in the ECG; P, Q, R, S and T.
The origin of the names of these waves is a matter of some
controversy~\cite{Hurst1998}, but whatever their origin, they are now enshrined
in the literature.
A schematic representation of the ECG waves is shown in
figure~\ref{fig:intro:ecg:schematic}.
Each of the waves is the result of the electrical activity in a particular part
of the heart.
Positive deflections are those which are above the baseline and negative ones
below.

The P-wave is caused by the depolarisation of the atria.
It has a relatively low amplitude because the atria are small and thin walled
compared to the ventricles, so there are not that many cells which can generate
the wave.
It tends to last from \ms{100} to \ms{120}.

The QRS complex is associated with the ventricular depolarisation.
It is a collection of up to three waves.
Any negative deflection which preceeds the R wave is the Q wave.
The R wave is the first positive deflection.
The S wave is the first negative deflection after the R wave.
A QRS complex does not need to have all three of the QRS waves present.
The QRS complex tends to have the largest magnitude in the ECG and lasts
approximately \ms{100}.

The T wave is associated with the ventricular repolarisation.
It occurs some time after the QRS complex.
The $\text{T}_{\text{P}}$, caused by the atrial repolarisation is not normally
visible on the ECG for a number of reasons.
It is very small in magnitude, it is also often masked either by the QRS complex
or by so called `baseline correction' algorithms which use the PR interval to
determine a `zero' for the ECG.

The axis of a wave is direction in which it has maximum amplitude.
This is determined using the hexaxial reference system.
A normal QRS complex (\ref{Lipman1994,Katz2006}) has an axis between \degr{-30}\
and \degr{+110}.
A normal P wave has an axis between \degr{+0}\ and \degr{+90}.


\subsection{The Frank Vectorcardiogram}


