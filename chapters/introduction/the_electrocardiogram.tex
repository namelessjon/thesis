\section{The Electrocardiograph}

The electrocardiograph, or ECG, was developed by Einthoven and colleagues at the
turn of the 20th century.
The Einthoven ECG used the string galvanometer, developed by Einthoven, to
record the potential differences between three sets of electrodes, or leads.
These electrodes were placed on each arm and on the left leg.
These three electrodes form the bases of many ECGs recorded to this day.
The string galvanometer was highly sensitive electrical recording device for its
time, and was developed by Einthoven.
For the development of the ECG and the string galvanometer, Einthoven was awarded
the nobel prize in 1924~\cite{Kligfield2002,TEXTBOOKS}.

Since Einthoven's day, the ECG has been continually refined.
This refinement has included both improvements in the way that individual leads
are measured, and new leads that are recorded.
In addition, there have been a number of specialised lead sets developed for
particular purposes, such as exercise recording and 24 hour measurement.
Research into new lead sets, to better understand the functioning of the heart,
continues to this day~\cite{PAPERS}.

\subsection{Lead Theory}

In Einthoven's original conception of the heart and the electrical field it
produced, he developed the `Einthoven Triangle'.
The Einthoven triangle is an equilateral triangle.
At its corners sit the three electrodes and the sides of the triangles are the
leads themselves.
At the centroid of the triangle sits the heart, which is represented by a
single, stationary, time varying dipole.
The potentials measured at the three electrodes are the potentials assuming the
system is two dimensional, homogeneous and infinite in extent.
This is not entirely incorrect.

Modern lead theory was developed in the 1950's.
In a series of three papers McFee and
Johnston~\cite{McFee1953,McFee1954a,McFee1954b}\ set out their concept of the
`lead field' which is still considered applicable today.
This theory, which was a further generalisation of work by Burger and van
Milaan, allows for a heart which consisted of distributed dipole sources
sitting in a three dimensional, finite and inhomogeneous medium.

First, it is useful to define what a lead is.
A lead is defined as a pair of terminals, connected to any number of electrodes
on the body, either directly or through any number of resistors or amplifiers.
One of the terminals is designated (arbitrarily) as the `positive' terminal.
The terminals are connected such that a positive measurement is taken, should
the `positive' terminal have a higher potential than the `negative' one.

The lead theory is based on the fundamental principles of linear volume
conductors set out by Helmholtz in the middle of the nineteenth century.
The first principle, that of superposition, states that the electric field
resulting from several sources in the medium is equal to the sum of the fields
which would be produced by each source considered alone.

The second principle, that of reciprocity, concerns current flow in the
medium.
It states that the current flow between two electrodes evoked by a field source in the
medium is the same as the current flow through the source evoked by placing a
potential difference across the two electrodes equal to the potential difference
that would have been created by the field source.
That is to say that the current flow is independent of the direction of
energisation, from within or outside the medium.

The concept of the lead field comes out of these principles.
To derive the lead field, we consider a unit current flowing in a lead.
The resulting flow of current through the body will have a certain magnitude and
direction at every point; it is a vector field.
Calling the vector field evoked by the unit current \vec{J}.

\subsection{The Twelve Lead Electrocardiogram}

\subsection{The Frank Vectorcardiogram}


