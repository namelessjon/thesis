\section{The Forward Problem}

The so-called `forward problem' seeks to find a relationship between the
electrical activity in heart and the potentials observed outside the heart,
most usefully on the surface of the body.
When looked at another way, it is a method of finding the lead field, $\vec{L}$,
for a given body.
The problem has been approached in a number of ways; experimentally, clinically
and numerically.

\subsection{Uses of the Forward Problem}

The original investigations into the forward problem very much concerned
themselves with finding $\vec{L}$.
They sought to refine the understanding of such concepts as the einthoven
triangle and the precordial electrodes and to more closely relate the differing
potentials observed with what was happening in the heart.
The experiments in the forward problem led to the lead field theory of McFee and
Johnson~\cite{McFee1953}\ and the Frank vectorcardiogram~\cite{Frank1956}.

Modern investigators of the forward problem use the solutions to perform similar
investigations, such as the development of new lead systems~\cite{Ihara2007}.
Forward solutions are also used for investigations into ischeamia, ventricular
and atrial fibrillation, conduction defects, amongst many.
Forward solutions also form the basis, or are used in the refinement, of inverse
solutions.

\subsection{A Brief History of the Forward Problem}

As has been mentioned, the initial uses of the forward problem were to determine
how the activity of the heart related to the leads.
These initial models tended to be real and physical models.
Towards this end, one of the first models constructed was by Burger and van
Milaan~\cite{Burger1946}, who constructed a one third life-size model of the
torso, filled with electrolyte solution.
The model had a cork spine and sandbags to represent the lungs.
The measurements from this model formed the basis for their refinements to the
lead theory.

Of the early mathematical models Brody~\cite{Brody1956}\ is one of the most
significant.
His was a model of the influence of the highly conductive blood masses of the
heart on the measured electrical field.
The `Brody Effect' is still used in literature today.

In the 1960s, numerical models of the torso started to
appear~\cite{Barr1966,Barnard1967}.
These initial models generally had a handful of dipoles, or even just one, to
represent the heart.
These were used to investigate the influence of dipole position and
heterogeneities on the body surface potential.

In 1971 Rush~\cite{Rush1971} published his torso model.
This was probably the ultimate torso model, literally and figuratively.
The model was twice lifesize and incorporated internal inhomogeneities
representing lungs, heart, heart blood, great blood vessels, liver, fat,
skeleton and anisotropic skeletal muscle--though the model could be used without
them as well.
The cardiac activity was represented by a number of dipole sources which were
located in the heart region of the torso.

In the 1980s, models of the heart and the torso gained more
sophistication~\cite{Gulrajani1983,Horacek1987}, though they were still generally
ventricular models, not models of the whole heart.
Models of this era routinely used 10s of dipole sources.
These models still did not include even primitive electrophysiology models,
although towards the end of the decade, propagation of the excitation wavefront
was being modeled~\cite{Aoki1987}.

Moving to the nineties and the turn of the millenium, the availability of
medical imaging tools such as CT and MRI scans allowed more accurate models to
be constructed~\cite{Weixue1993,Weixue1996}.
This is also when the first models began to appear which incorporated the
atrium~\cite{Weixue1996,vanDam2005}.
Many of the models published in this era use myocyte models to simulate the
propagation, some even using biophysically detailed models.

\subsection{Numerical Approaches}

Numerical solutions to the forward problem involve solving Maxwell's equations
within the torso.
The same assumptions which were made for the lead field theory
($\S$\ref{sec:intro:ecg:lead_theory}) are valid here.
To reiterate them briefly, they state that the body is a linear volume
conductor.
There are no inductive or capacitive effects.
Furthermore, due to the finite (and small) size of the body, there are no
propagation effects to consider.
The field in the body, $\vec{E}$ is therefore given by
\begin{equation}
\label{eqn:intro:forward:maxwell}
\vec{E} = -\nabla\phi
\end{equation}
where $\phi$ is a scalar potential field.
Current flow in the torso under these conditions is given by Ohms law, which
states the total current flow, $\vec{J}$, is
\begin{equation}
\label{eqn:intro:forward:ohm}
\vec{J} = \mathbf{\sigma}\vec{E} + \vec{J^i}
\end{equation}
where $\mathbf{\sigma}$ is the conductivity tensor and $\vec{J^i}$ is an
impressed, or applied, current density which is generated by active sources,
i.e. the heart.

Since the total current flow is solenoidal, i.e. the net current flow into and
out of a volume is zero, via (\ref{eqn:intro:forward:ohm}) we have that
\begin{equation}
\label{eqn:intro:forward:ohm2}
\nabla\cdot\vec{J} = 0 = \nabla\cdot\left(\mathbf{\sigma}\vec{E} + \vec{J^i} \right)
\end{equation}
This can alternatively be written as
\begin{equation}
\label{eqn:intro:forward:poisson}
\nabla\cdot\vec{J^i} = \mathbf{\sigma}\nabla^2\phi
\end{equation}
If the body is homogeneous and isotropic, it becomes
\begin{equation}
\label{eqn:intro:forward:poisson2}
\nabla^2\phi = \frac{\nabla\cdot\vec{J^i}}{\sigma}
\end{equation}
which is Poisson's equation.

Finding $\phi$ for a given $\vec{J^i}$ can be achieved using a volume or
boundary based approach.
Volume based approaches (\cite{Seger2004,Klepfer1997,Keller2007}) involve dividing the body
up into volumes of different conductivity.
The solution for $\phi$ is found using a finite element or finite difference
method.
Boundary element based approaches
(\cite{Barr1966,Clayton2002,Gulranjani1989,Weixue1996}) divide the torso into
regions of isotropic and uniform conductivity.
The potential, $\phi$, is found on the surfaces of these regions.

Volume based methods allow for a more complex model.
Interior volumes can have internally varying or even anisotropic conductivity.
They tend to be much more computationally intensive to solve, however, because
the problem space is still three dimensional.
In addition, the required element size can be quite small, especially for finite
difference approximations.
Keller~\cite{Keller2007}\ used a model with more than ten million torso
elements, for example, whilst the finite element model used by
Klepfer~\cite{Klepfer1997}\ has approximately one million elements and 168,000
nodes.

Boundary element based methods by contrast tend to be much simpler to solve.
Typical model sizes are of the order of a thousand to ten thousand elements, as
they reduce the problem space to a series of two dimensional surfaces.
This reduces the computational effort required.
However, since the solution is only computed at the surfaces, regions can only
be uniform and isotropic.





